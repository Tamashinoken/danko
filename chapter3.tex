\chapter{Fazoda analitik geometriya masalalari}
\section{Tekislik va to`g`ri chiziq}
\textbf{Tekislik}.
Tekislik tenglamasining vektor ko`rinishi quyidagicha bo`ladi:
$$\textbf{r}\cdot\textbf{n}=p$$
Bu yerda $\textbf{r}=x\textbf{i}+y\textbf{j}+z\textbf{k}$ -- tekislikda joylashgan $M(x,y,z)$ nuqtaning radius-vektori; $\textbf{n}=\textbf{i}\cos\alpha+\textbf{j}\cos\beta+\textbf{k}\cos\gamma$ -- tekislikka koordinatalar boshidan tushirilgan perpendikulyar bo`ylab yo`nalgan birlik vektor; $\alpha$, $\beta$, $\gamma$ -- $Ox$, $Oy$, $Oz$ o`qlari bilan ushbu vektor hosil qilgan burchaklar; $p$ -- ushbu perpendikulyarning uzunligi.

Koordinatalar sistemasiga o`tganda yuqoridagi formula quyidagi ko`rinishga keladi:
\begin{equation}x\cos\alpha+ y\cos\beta+z\cos\gamma\end{equation}

Ixtiyoriy tekislik tenglamasini quyidagi ko`rinishda ham yozish mumkin:
\begin{equation}
	Ax+By+Cz+D=0
\end{equation}
Buning uchun $A^2+B^2+C^2\ne0$ shart bajarilishi kerak. Bu yerda $A$, $B$, $C$ larni tekislikka perpendikulyar bo`lgan $\textbf{N}=A\textbf{i}+B\textbf{j}+C\textbf{k}$ vektorning koordinatalari sifatida qarash mumkin. Tekislikning umumiy tenglamasini normal ko`rinishga keltirish uchun tenglama barcha hadlarini normallashtiruvchi ko`paytuvchiga ko`paytirish kerak:
\begin{equation}
	\mu=\pm1/N=\pm\frac{1}{\sqrt{A^2+B^2+C^2}}
\end{equation}
bu yerda kvadrat ildizdan oldinda turgan belgi umumiy tenglamadagi $D$ ning ishorasiga qarama-qarshi qilib olinadi.

$Ax+By+Cz+D=0$ tenglama bilan aniqlanadigan tekislik uchun xususiy hollar:
\begin{itemize}
	\item $A=0$; tekislik $Ox$ o`qiga parallel
	\item $B=0$; tekislik $Oy$ o`qiga parallel
	\item $C=0$; tekislik $Oz$ o`qiga parallel
	\item $D=0$; tekislik koordinatalar boshidan o`tadi
	\item $A=B=0$; tekislik $Oz$ o`qiga perpendikulyar ($xOy$ tekislikka parallel)
	\item $A=C=0$; tekislik $Oy$ o`qiga perpendikulyar ($xOz$ tekislikka parallel)
	\item $B=C=0$; tekislik $Ox$ o`qiga perpendikulyar ($yOz$ tekislikka parallel)
	\item $A=D=0$; tekislik $Ox$ o`qi orqali o`tadi
	\item $B=D=0$; tekislik $Oy$ o`qi orqali o`tadi
	\item $C=D=0$; tekislik $Oz$ o`qi orqali o`tadi
\end{itemize}
\section{Ikkinchi tartibli sirtlar}
