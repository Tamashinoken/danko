\chapter{Fazoda analitik geometriya masalalari}
\section{Tekislik va to`g`ri chiziq}
\textbf{Tekislik}.
Tekislik tenglamasining vektor ko`rinishi quyidagicha bo`ladi:
$$\textbf{r}\cdot\textbf{n}=p$$
Bu yerda $\textbf{r}=x\textbf{i}+y\textbf{j}+z\textbf{k}$ -- tekislikda joylashgan $M(x,y,z)$ nuqtaning radius-vektori; $\textbf{n}=\textbf{i}\cos\alpha+\textbf{j}\cos\beta+\textbf{k}\cos\gamma$ -- tekislikka koordinatalar boshidan tushirilgan perpendikulyar bo`ylab yo`nalgan birlik vektor; $\alpha$, $\beta$, $\gamma$ -- $Ox$, $Oy$, $Oz$ o`qlari bilan ushbu vektor hosil qilgan burchaklar; $p$ -- ushbu perpendikulyarning uzunligi.

Koordinatalar sistemasiga o`tganda yuqoridagi formula quyidagi ko`rinishga keladi:
\begin{equation}x\cos\alpha+ y\cos\beta+z\cos\gamma\end{equation}

Ixtiyoriy tekislik tenglamasini quyidagi ko`rinishda ham yozish mumkin:
\begin{equation}
	Ax+By+Cz+D=0
\end{equation}
Buning uchun $A^2+B^2+C^2\ne0$ shart bajarilishi kerak. Bu yerda $A$, $B$, $C$ larni tekislikka perpendikulyar bo`lgan $\textbf{N}=A\textbf{i}+B\textbf{j}+C\textbf{k}$ vektorning koordinatalari sifatida qarash mumkin. Tekislikning umumiy tenglamasini normal ko`rinishga keltirish uchun tenglama barcha hadlarini normallashtiruvchi ko`paytuvchiga ko`paytirish kerak:
\begin{equation}
	\mu=\pm1/N=\pm\frac{1}{\sqrt{A^2+B^2+C^2}}
\end{equation}
bu yerda kvadrat ildizdan oldinda turgan belgi umumiy tenglamadagi $D$ ning ishorasiga qarama-qarshi qilib olinadi.

$Ax+By+Cz+D=0$ tenglama bilan aniqlanadigan tekislik uchun xususiy hollar:
\begin{itemize}
	\item $A=0$; tekislik $Ox$ o`qiga parallel
	\item $B=0$; tekislik $Oy$ o`qiga parallel
	\item $C=0$; tekislik $Oz$ o`qiga parallel
	\item $D=0$; tekislik koordinatalar boshidan o`tadi
	\item $A=B=0$; tekislik $Oz$ o`qiga perpendikulyar ($xOy$ tekislikka parallel)
	\item $A=C=0$; tekislik $Oy$ o`qiga perpendikulyar ($xOz$ tekislikka parallel)
	\item $B=C=0$; tekislik $Ox$ o`qiga perpendikulyar ($yOz$ tekislikka parallel)
	\item $A=D=0$; tekislik $Ox$ o`qi orqali o`tadi
	\item $B=D=0$; tekislik $Oy$ o`qi orqali o`tadi
	\item $C=D=0$; tekislik $Oz$ o`qi orqali o`tadi
	\item $A=B=D=0$; tekislik $xOy$ tekislik bilan mos tushadi ($z=0$)
	\item $A=C=D=0$; tekislik $xOz$ tekislik bilan mos tushadi ($y=0$)
	\item $B=C=D=0$; tekislik $yOz$ tekislik bilan mos tushadi ($x=0$)
\end{itemize}
Agar tekislikning umumiy tenglamasida $D\ne0$ bo`lsa, tenglamaning barcha hadlarini $-D$ ga bo`lib, quyidagi ko`rinishga keltirish mumkin:
\begin{equation}
	\frac{x}{a}+\frac{y}{b}+\frac{z}{c}=1
	\label{3.4}
\end{equation}
bu yerda $a=-D/A$, $b=-D/B$, $c=-D/C$. \eqref{3.4}-tenglamani tekislikning kesmalardagi tenglamasi deb ataladi. Undagi $a$, $b$, $c$ lar mos holda tekislik $Ox$, $Oy$ va $Oz$ o`qlari bilan kesishish nuqtalarining absissa, ordinata va aplikatalari.



\begin{enumerate}\setcounter{enumi}{295}
	\item Quyidagi tekisliklar tenglamalarini normal ko`rinishga keltiring:
	1)$x+y-z-2=0$; \ \ 2)$3x+5y-4z+7=0$
	\item $M_0(1;3;-2)$ nuqtadan $2x-3y-4z+12=0$ tekislikkacha bo`lgan masofani hisoblang. $M_0$ nuqta tekislikka nisbatan qanday joylashgan?
	\item $M_0(2;3;-5)$ nuqtadan $4x-2y+5z-12=0$ tekislikka tushirilgan perpendikulyar uzunligini hisoblang.
	\item 1) $M(-2;3;4)$ nuqtadan o`tib, koordinata o`qlarida teng uzunlikdagi kesmalar ajratuvchi tekislik tenglamasini toping;
	2) $N(2;-1;4)$ nuqtadan o`tib, $Oz$ o`qida $Ox$ va $Oy$ o`qlariga nisbatan ikki marta uzun kesma ajratadigan tekislik tenglamasini toping.
	\item $P(2;0;-1)$ va $Q(1;-1;3)$ nuqtalardan o`tib, $3x+2y-z+5=0$ tekislikka perpendikulyar bo`lgan tekislik tenglamasini toping. 
	\item $2x-5y+2z+5=0$ tekislikda shunday $M$ nuqtani topingki, $OM$ to`g`ri chiziq koordinata o`qlarida bir xil burchaklar hosil qilsin.
	\item $P(4;-3;12)$ nuqta, koordinatalar boshidan tekislikka tushirilgan perpendikulyar asosi bo`lsa, ushbu tekislik tenglamasini toping.
	\item $3x-4y+5z-12=0$ tekislikka perpendikulyar bo`lgan tekisliklar tenglamalarini toping.
	\item $P(1;-4;2)$ va $Q(7;1;-5)$ nuqtalardan bir xil masofada joylashgan tekislik tenglamasini toping.
	

	
	
\end{enumerate}
\section{Ikkinchi tartibli sirtlar}
