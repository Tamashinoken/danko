\chapter{Fazoda analitik geometriya masalalari}
\section{Tekislik va to`g`ri chiziq}
\textbf{Tekislik}.
Tekislik tenglamasining vektor ko`rinishi quyidagicha bo`ladi:
$$\textbf{r}\cdot\textbf{n}=p$$
Bu yerda $\textbf{r}=x\textbf{i}+y\textbf{j}+z\textbf{k}$ -- tekislikda joylashgan $M(x,y,z)$ nuqtaning radius-vektori; $\textbf{n}=\textbf{i}\cos\alpha+\textbf{j}\cos\beta+\textbf{k}\cos\gamma$ -- tekislikka koordinatalar boshidan tushirilgan perpendikulyar bo`ylab yo`nalgan birlik vektor; $\alpha$, $\beta$, $\gamma$ -- $Ox$, $Oy$, $Oz$ o`qlari bilan ushbu vektor hosil qilgan burchaklar; $p$ -- ushbu perpendikulyarning uzunligi.

Koordinatalar sistemasiga o`tganda yuqoridagi formula quyidagi ko`rinishga keladi:
\begin{equation}x\cos\alpha+ y\cos\beta+z\cos\gamma\end{equation}

Ixtiyoriy tekislik tenglamasini quyidagi ko`rinishda ham yozish mumkin:
\begin{equation}
	Ax+By+Cz+D=0
\end{equation}
Buning uchun $A^2+B^2+C^2\ne0$ shart bajarilishi kerak. 
\section{Ikkinchi tartibli sirtlar}
