\chapter{Matritsa va determinantlar}
\section{n-tartibli determinant haqida tushuncha}
$$
\begin{pmatrix}
	a_{11}&a_{12}&a_{13}&a_{14}\\
	a_{21}&a_{22}&a_{23}&a_{24}\\
	a_{31}&a_{32}&a_{33}&a_{34}\\
	a_{41}&a_{42}&a_{43}&a_{44}\\
\end{pmatrix}
$$ ko`rinishidagi to`rtinchi tartibli determinant quyidagi ifoda orqali hisoblanadi:
\begin{equation}
	\begin{vmatrix}
		a_{11}&a_{12}&a_{13}&a_{14}\\
		a_{21}&a_{22}&a_{23}&a_{24}\\
		a_{31}&a_{32}&a_{33}&a_{34}\\
		a_{41}&a_{42}&a_{43}&a_{44}\\
	\end{vmatrix}=a_{11}
\begin{vmatrix}
	a_{22}&a_{23}&a_{24}\\
	a_{32}&a_{33}&a_{34}\\
	a_{42}&a_{43}&a_{44}\\
\end{vmatrix}-a_{12}
\begin{vmatrix}
	a_{21}&a_{23}&a_{24}\\
	a_{31}&a_{33}&a_{34}\\
	a_{41}&a_{43}&a_{44}\\
\end{vmatrix}+a_{13}
\begin{vmatrix}
	a_{21}&a_{22}&a_{24}\\
	a_{31}&a_{32}&a_{34}\\
	a_{41}&a_{42}&a_{44}\\
\end{vmatrix}-a_{14}
\begin{vmatrix}
	a_{21}&a_{22}&a_{23}\\
	a_{31}&a_{32}&a_{33}\\
	a_{41}&a_{42}&a_{43}\\
\end{vmatrix}	
\label{4.1}	
\end{equation}

4-tartibli determinantlar yordamida 5- va undan yuqori tartibli determinant tushunchalarini ham kiritish mumkin.

Ixtiyoriy tartibli determinantlar uchun ma'lum bir element uchun algebraik to`liruvchi va minorlarni aniqlash masalasi o`z kuchida qoladi. Shuningdek, uchinchi tartibli determinantlar uchun keltirilgan algebraik to`ldiruv-chilar haqidagi  ikkita teorema yuqori tartibli determinantlar uchun ham o`rinli bo`ladi.

Shunday qilib, $M_{ik}$ orqali minorni, $A_{ik}$ orqali esa $n$-tartibli determinant $a_{ik}$ elementi (ya'ni, $i$-satr va $k$-ustunda joylashgan element)ning algebraik to`ldiruvchisini ifodalagan holda, quyidagi ifodani hosil qilish mumkin:
\begin{equation}
	A_{ik}=(-1)^{i+k}M)_{ik}
	\label{4.2}
\end{equation}
$D$ -- $n$ tartibli determinant bo`lsin. Uni dastlab $i$-satr elementlari bo`yicha, so`ngra esa $k$-ustun elementlari bo`yicha yoyib chiqqan holda, quyidagini hosil qilamiz:
\begin{subequations}
\begin{equation}
	D=a_{i1}A_{i1}+a_{i2}A_{i2}+\ldots +a_{in}A_{in};
	\label{4.3a}
\end{equation}
\begin{equation}
	D=a_{1k}A_{1k}+a_{2k}A_{2k}+\ldots+a_{nk}A_{nk}
\label{4.3b}
\end{equation}
\end{subequations}
Boshqa tomondan, $j\ne i$ hamda $k\ne l$ bo`lganda:
\begin{subequations}
	\begin{equation}
		a_{j1}A_{i1}+a_{j2}A_{i2}+\ldots+a_{jn}A_{in}=0;
		\label{4.4a}
	\end{equation}
\begin{equation}
	a_{1k}A_{1l}+a_{2k}A_{2l}+\ldots+a_{nk}A_{nl}=0
	\label{4.4b}
\end{equation}
\end{subequations}
Ikkinchi va uchinchi tartibli determinantlar uchun keltirilgan xossalar ixtiyoriy tartibli determinantlar uchun ham o`rinli bo`ladi.

$$
\begin{cases} 
	a_{11}x_{1}+a_{12}x_{2}+\ldots+a_{1n}x_{n}=b_{1} \\ 
	a_{21}x_{1}+a_{22}x_{2}+\ldots+a_{2n}x_{n}=b_{2} \\ 
	\ldots \ldots\ldots\ldots\ldots\ldots\ldots\ldots\ldots\ldots\\
	a_{n1}x_{1}+a_{n2}x_{2}+\ldots+a_{nn}x_{n}=b_{n}
\end{cases} 
$$
ko`rinishidagi chiziqli tenglamalar sistemasini yechish uchun:
$$
D=
\begin{vmatrix}
	a_{11}&a_{12}&\ldots&a_{1n}\\
	a_{21}&a_{22}&\ldots&a_{2n}\\
	\ldots&\ldots&\ldots&\ldots\\
	a_{n1}&a_{n2}&\ldots&a_{nn}\\
\end{vmatrix}\ne 0
$$ 
determinantni topish zarur bo`ladi. Tenglamalarning yechimlari esa quyidagi ko`rinishda topiladi:
\begin{equation*}
	x{1}=\frac{D_{1}}{D};\ \ \ \ x_{2}=\frac{D_{2}}{D};\ldots \ \ \ x_{k}=\frac{D_{k}}{D}
\end{equation*} 
bu yerda: $D$ - matritsaning asosiy determinanti, $D_{k}$ ($k=1,2,\ldots,n$) esa matritsa $k$-ustunini ozod hadlar bilan almashtirish natijasida hosil bo`lgan matritsalarning determinantlari:
\begin{equation*}
	D_{k}=\begin{vmatrix}
		a_{11}&a_{12}&\ldots&a_{1,k-1}&b_{1}&a_{1,k+1}&\ldots&a_{1n}\\
		a_{21}&a_{22}&\ldots&a_{2,k-1}&b_{2}&a_{2,k+1}&\ldots&a_{2n}\\
		\ldots&\ldots&\ldots&\ldots&\ldots&\ldots&\ldots&\ldots\\
		a_{n1}&a_{n2}&\ldots&a_{n,k-1}&b_{n}&a_{n,k+1}&\ldots&a_{nn}\\
	\end{vmatrix}
\end{equation*}


\begin{enumerate}
	\setcounter{enumi}{382}
	\item Determinantni hisoblang:
	$$\begin{vmatrix}
		3&5&7&2\\
		1&2&3&4\\
		-2&-3&3&2\\
		1&3&5&4\\	
	\end{vmatrix}$$ 

$\triangle$ Misolni yechish tartibi quyidagicha: 

1) 2-satr elementlarining hammasini 3 ga ko`paytirib, 1-satrdan ayirib tashlaymiz; 

2) 3-satr elementlariga ikkilangan 2-satr elementlarini qo`shamiz; 

3) 4-satr elementlaridan 2-satr elementlarini ayirib tashlaymiz.

 Natijada determinant quyidagi ko`rinishga  keladi:
$$D=\begin{vmatrix}
	0&-1&-2&-10\\
	1&2&3&4\\
	0&1&9&10\\
	0&1&2&0\\
\end{vmatrix}$$
Ushbu determinantni 1-ustun elementlari bo`yicha yoyib chiqamiz:
$$D=-\begin{vmatrix}
	-1&-2&-10\\
	1&9&10\\
	1&2&0\\
\end{vmatrix}$$

1-satr elementlariga 3-satr elementlarini qo`shib natijani 1-satrga, 2-satr elementlaridan 3-satr elementlarini ayirib natijani 2-satrga va 3-satr elementlarini o`z holicha 3-satrga yozamiz:
$$D=-\begin{vmatrix}
	0&0&-10\\
	0&7&10\\
	1&2&0\\
\end{vmatrix}$$
Hosil bo`lgan determinantni 1-ustun elementlari bo`yicha yoyib chiqamiz:
$$D=-\begin{vmatrix}
	0&-10\\
	7&10\\
\end{vmatrix}=-70.\ \blacktriangle$$

\item Determinantni hisoblang:
$$\begin{vmatrix}
	1&2&0&0&0\\
	3&2&3&0&0\\
	0&4&3&4&0\\
	0&0&5&4&5\\
	0&0&0&6&5\\
\end{vmatrix}$$

$\triangle$ 2-, 4- va 5-ustunlardan umumiy ko`paytuvchini determinant belgisidan tashqariga chiqaramiz:
$$D=2\cdot2\cdot5\cdot\begin{vmatrix}
	1&1&0&0&0\\
	3&1&3&0&0\\
	0&2&3&2&0\\
	0&0&5&2&1\\
	0&0&0&3&1\\
\end{vmatrix}$$

2-ustun elementlaridan 1-ustun elementlarini ayirib tashlaymiz va hosil bo`lgan determinantni 1-satr elementlari bo`yicha yoyib chiqamiz:
$$D=20\cdot\begin{vmatrix}
	1&0&0&0&0\\
	3&-2&3&0&0\\
	0&2&3&2&0\\
	0&0&5&2&1\\
	0&0&0&3&1\\
\end{vmatrix}=20\cdot\begin{vmatrix}
-2&3&0&0\\
2&3&2&0\\
0&5&2&1\\
0&0&3&1
\end{vmatrix}$$

2-satr elementlariga 1-satr elementlarini qo`shamiz, $-2$ (1-ustundagi umumiy ko`paytuvchi)ni determinant belgisidan tashqariga chiqaramiz. So`ng hosil bo`lgan determinantni 1-ustun elementlari bo`yicha yoyib chiqamiz:
$$D=-40\cdot\begin{vmatrix}
	1&3&0&0\\
	0&6&2&0\\
	0&5&2&1\\
	0&0&3&1
\end{vmatrix}=-40\cdot\begin{vmatrix}
6&2&0\\
5&2&1\\
0&3&1
\end{vmatrix}$$

2-satr elementlaridan 3-satr elementlarini ayiramiz, 2 (1-satrdagi umumiy ko`paytuvchi)ni determinant belgisidan tashqariga chiqaramiz. Hosil bo`lgan determinantni 3-ustun elementlari bo`yicha yoyib chiqamiz:
$$D=-80\cdot\begin{vmatrix}
	3&1&0\\
	5&-1&0\\
	0&3&1
\end{vmatrix}=-80\cdot\begin{vmatrix}
3&1\\
5&-1\\
\end{vmatrix}=640.\ \blacktriangle$$

\item Quyidagi tenglamalar sistemasidan $y$ ni aniqlang:
$$\begin{cases}
	x+2y+3z=14\\
	y+2z+3t=20\\
	z+2t+3x=14\\
	t+2x+3y=12
\end{cases}
	$$
	
	$\triangle$ Dastlab determinantni hisoblaymiz:
	$$D=\begin{vmatrix}
		1&2&3&0\\
		0&1&2&3\\
		3&0&1&2\\
		2&3&0&1\\
	\end{vmatrix}$$
2-ustun elementlaridan ikkilangan 1-ustun elementlarini ayirib tashlaymiz; 3-ustun elementlaridan uchlangan 1-ustun elementlarini ayirib tashlaymiz:
$$D=\begin{vmatrix}
	1&0&0&0\\
	0&1&2&3\\
	3&-6&-8&2\\
	2&-1&-6&1\\
\end{vmatrix}=\begin{vmatrix}
1&2&3\\
-6&-8&2\\
-1&-6&1\\
\end{vmatrix}=(-2)\cdot(-1)\cdot\begin{vmatrix}
1&2&3\\
3&3&-1\\
1&6&-1\\
\end{vmatrix}$$
Yana bir  marta 2-ustun elementlaridan ikkilangan 1-ustun elementlarini ayirib tashlaymiz; 3-ustun elementlaridan uchlangan 1-ustun elementlarini ayirib tashlaymiz:
$$D=2\begin{vmatrix}
	1&0&0\\
	3&-2&-10\\
	1&4&-4\\
\end{vmatrix}=2\cdot\begin{vmatrix}
-2&-10\\
4&-4\\
\end{vmatrix}=2\cdot(8+40)=96$$
Endi esa:
$$D_{y}=\begin{vmatrix}
	1&14&3&0\\
	0&20&2&3\\
	3&14&1&2\\
	2&12&0&1\\
\end{vmatrix}=2\cdot\begin{vmatrix}
1&7&3&0\\
0&10&2&3\\
3&7&1&2\\
2&6&0&1\\
\end{vmatrix}$$
3-satr elementlaridan uchlangan 1-satr elementlarini ayirib tashlaymiz; 4-satr elementlaridan ikkilangan 1-satr elementlarini ayirib tashlaymiz:
$$D_{y}=2\cdot\begin{vmatrix}
	1&7&3&0\\
	0&10&2&3\\
	0&-14&-8&2\\
	0&-8&-6&1\\
\end{vmatrix}=2\cdot\begin{vmatrix}
10&2&3\\
-14&-8&2\\
-8&-6&1\\
\end{vmatrix}=2\cdot2\cdot2\cdot\begin{vmatrix}
5&1&3\\
-7&-4&2\\
-4&-3&1\\
\end{vmatrix}$$
1-satr elementlaridan uchlangan 3-satr elementlarini ayirib tashlaymiz; 2-satr elementlaridan ikkilangan 3-satr elementlarini ayirib tashlaymiz:
$$D_{y}=\begin{vmatrix}
	17&10&0\\
	1&2&0\\
	-4&-3&1\\
\end{vmatrix}=8\cdot\begin{vmatrix}
17&10\\
1&2\\
\end{vmatrix}=192$$
Demak, $y=D_{y}/D=192/96=2.\ \blacktriangle$

\item Determinantni hisoblang:
$$V=\begin{vmatrix}
	1&1&1&1\\
	a&b&c&d\\
	a^{2}&b^{2}&c^{2}&d^{2}\\
	a^{3}&b^{3}&c^{3}&d^{3}\\
\end{vmatrix}$$

$\triangle$ 1)\ 1-satr elementlarini $a$ ga ko`paytirib, 2-satr elementlaridan ayirib tashlaymiz; 

2)\ 2-satr elementlarini $a$ ga ko`paytirib, 3-satr elementlaridan ayirib tashlaymiz; 

3)\ 3-satr elementlarini $a$ ga ko`paytirib, 4-satr elementlaridan ayirib tashlaymiz:
$$V=\begin{vmatrix}
	1&1&1&1\\
	0&b-a&c-a&d-a\\
	0&b^{2}-ab&c^{2}-ac&d^{2}-ad\\
	0&b^{3}-ab^{2}&c^{3}-ac^{2}&d^{3}-ad^{2}\\
\end{vmatrix}=(b-a)(c-a)(d-a)
\begin{vmatrix}
1&1&1\\
b&c&d\\
b^{2}&c^{2}&d^{2}\\
\end{vmatrix}$$
1-satr elementlarini $b$ ga ko`paytirib 2-satr elementlaridan ayiramiz; 2-satr elementlarini $b$ ga ko`paytirib 3-satr elementlaridan ayiramiz:
\begin{multline*}
	V=(b-a)(c-a)(d-a)\begin{vmatrix}
	1&1&1\\
	0&c-b&d-b\\
	0&c^{2}-bc&d^{2}-bd
\end{vmatrix}=(b-a)(c-a)(d-a)(c-b)(d-b)\begin{vmatrix}
1&1\\
c&d
\end{vmatrix}=\\=(b-a)(c-a)(d-a)(c-b)(d-b)(d-c).
\end{multline*}
Ko`rinib turibdiki, agar $a$, $b$, $c$ va $d$ larning  orasida teng qiymatlilari bo`lsagina $V$ determinant nolga teng bo`ladi. $\blacktriangle$
\end{enumerate}
Determinantlarni hisoblang:
\begin{enumerate}\setcounter{enumi}{386}
	\item $\begin{vmatrix}
		1&-2&3&4\\
		2&1&-4&3\\
		3&-4&-1&2\\
		4&3&2&-1
	\end{vmatrix}$
	\inlineitem
	$\begin{vmatrix}
		-1&-1&-1&-1\\
		-1&-2&-4&-8\\
		-1&-3&-9&-27\\
		-1&-4&-16&-64
	\end{vmatrix}$
\item$\begin{vmatrix}
	10&2&0&0&0\\
	12&10&2&0&0\\
	0&12&10&2&0\\
	0&0&12&10&2\\
	0&0&0&12&10
\end{vmatrix}$
	\inlineitem
	$\begin{vmatrix}
		1+a&1&1&1\\
		1&1-a&1&1\\
		1&1&1+b&1\\
		1&1&1&1-b\\
	\end{vmatrix}$
\end{enumerate}	

Tenglamalar sistemasini yeching:

\begin{enumerate}\setcounter{enumi}{390}
	\item $\begin{cases}
		y-3z+4t=-5\\
		x-2z+3t=-4\\
		3x+2y-5t=12\\
		4x+3y-5z=5
	\end{cases}$
\inlineitem
$\begin{cases}
	x-3y+5z-7t=12\\
	3x-5y+7z-t=0\\
	5x-7y+z-3t=4\\
	7x-y+3z-5t=16
\end{cases}$

\item $\begin{cases}
	x+2y=5\\
	3y+4z=18\\
	5z+6u=39\\
	7u+8v=68\\
	9v+10x=55
\end{cases}$
\inlineitem
$\begin{cases}
	2x+3y-3z+4t=7\\
	2x+y-z+2t=5\\
	6x+2y+z=4\\
	2x+3y-5t=-11
\end{cases}$

\end{enumerate}

\section{Chiziqli almashtirishlar va matritsalar}

$x$ va $y$ o`zgaruvchilarning qiymatlarini 
$$x=a_{11}x^\prime+a_{12}y^\prime$$
$$y=a_{21}x^{\prime}+a_{22}y^\prime$$
tenglamalar yordamida $x^\prime$ va $y^\prime$ o`zgaruvchilar orqali ifodalash mumkin. Ushbu tenglamalar $x^\prime$ va $y^\prime$ o`zgaruvchilarning chiziqli almashtirish tenglamalari deb ataladi. Ularni nuqta (yoki vektor)ning tekislikdagi koordinatalarini chiziqli almashtirish deb ham qarash mumkin. 

$$A=\begin{pmatrix}
	a_{11}&a_{12}\\
	a_{21}&a_{22}
\end{pmatrix}$$ ko`rinishidagi matritsani chiziqli almashtirish matritsasi deyiladi. 
$$D=\begin{vmatrix}
	a_{11}&a_{12}\\
	a_{21}&a_{22}
\end{vmatrix}$$ko`rinishidagi determinantni esa chiziqli almashtirish determinanti deyiladi. Bundan keyin faqat $D\ne0$ bo`lgan hollarni ko`rib chiqamiz.

Shuningdek, fazodagi \footnote[1]{Ba'zida chiziqli almashtirish deganda quyidagi umumiyroq ko`rinishdagi tengliklar tushuniladi:
	$$x=a_{11}x^\prime+a_{12}y^\prime+a_{13}z^\prime+b_1$$
	$$x=a_{21}x^\prime+a_{22}y^\prime+a_{23}z^\prime+b_2$$
	$$x=a_{31}x^\prime+a_{32}y^\prime+a_{33}z^\prime+b_3$$
Bu yerda $b_1=b_2=b_3=0$ bo`lgan hol uchun chiziqli almashtirishlar ko`rib chiqiladi. Funksional analiz kursida chiziqli almashtirishni shuningdek chiziqli operator deb ham atladi.} uchta o`zgaruvchi uchun ham chiziqli almashtirishlarni ko`rib chiqish mumkin:
$$x=a_{11}x^\prime+a_{12}y^\prime+a_{13}z^\prime$$
$$x=a_{21}x^\prime+a_{22}y^\prime+a_{23}z^\prime$$
$$x=a_{31}x^\prime+a_{32}y^\prime+a_{33}z^\prime$$
bu yerda 
$$A=\begin{pmatrix}
	a_{11}&a_{12}&a_{13}\\
	a_{21}&a_{22}&a_{23}\\
	a_{31}&a_{32}&a_{33}
\end{pmatrix}$$
va 
$$D=\begin{vmatrix}
	a_{11}&a_{12}&a_{13}\\
	a_{21}&a_{22}&a_{23}\\
	a_{31}&a_{32}&a_{33}
\end{vmatrix}$$
mos holda almashtirish matritsasi va determinanti.

Agar $D_A\ne0$ bo`lsa, $A$ matritsani \textcolor{red}{aynimagan(o`ziga xos bo`lmagan)} matritsa deyiladi, $D_A=0$ bo`lsa, \textcolor{red}{aynigan(o`ziga xos)} matritsa deyiladi.

$$\begin{pmatrix}
	a_{11}&a_{12}\\
	a_{21}&a_{22}
\end{pmatrix}\ \mbox{va}\ \ \begin{pmatrix}
a_{11}&a_{12}&a_{13}\\
a_{21}&a_{22}&a_{23}\\
a_{31}&a_{32}&a_{33}
\end{pmatrix}$$ ko`rinishidagi matritsalarni mos holda ikkinchi va uchinchi tartibli kvadrat matritsalar deb ata-ladi.

Bundan keyingi o`rinlarda  bir qator ta'riflar uchinchi tartibli matritsalar uchun beriladi; biroq ularni ikkinchi tartibli matritsalar uchun ham qo`llash qiyinchilik tug`dirmaydi.

Agar kvadrat matritsaning elementlari $a_{mn}=a_{nm}$ shartni qanoatlantirsa, bunday matritsalarni simmetrik matritsalar deb ataladi.

Quyidagi ikki matritsa
$$A=\begin{pmatrix}
	a_{11}&a_{12}&a_{13}\\
	a_{21}&a_{22}&a_{23}\\
	a_{31}&a_{32}&a_{33}
\end{pmatrix}\ \ \mbox{va}\ \ B=\begin{pmatrix}
b_{11}&b_{12}&b_{13}\\
b_{21}&b_{22}&b_{23}\\
b_{31}&b_{32}&b_{33}\\
\end{pmatrix}$$
faqatgina ularning mos elementlari teng bo`lganda, ya'ni $a_{mn}=b_{mn}\ (m,\ n=1,2,3,\ldots)$ shart bajarilgandagina teng ($A=B$) bo`ladi.

$A$ va $B$ matritsaning yig`indisi quyidagi tenglik orqali ifodalanadi:
$$\begin{pmatrix}
	a_{11}&a_{12}&a_{13}\\
	a_{21}&a_{22}&a_{23}\\
	a_{31}&a_{32}&a_{33}
\end{pmatrix}
	+\begin{pmatrix}
		b_{11}&b_{12}&b_{13}\\
		b_{21}&b_{22}&b_{23}\\
		b_{31}&b_{32}&b_{33}\\
	\end{pmatrix}=\begin{pmatrix}
	a_{11}+b_{11}&a_{12}+b_{12}&a_{13}+b_{13}\\
	a_{21}+b_{21}&a_{22}+b_{22}&a_{23}+b_{23}\\
	a_{31}+b_{31}&a_{32}+b_{32}&a_{33}+b_{33}
\end{pmatrix}
$$
$A$ matritsani $m$ songa ko`paytirish:
$$m\begin{pmatrix}
	a_{11}&a_{12}&a_{13}\\
	a_{21}&a_{22}&a_{23}\\
	a_{31}&a_{32}&a_{33}
\end{pmatrix}=\begin{pmatrix}
ma_{11}&ma_{12}&ma_{13}\\
ma_{21}&ma_{22}&ma_{23}\\
ma_{31}&ma_{32}&ma_{33}
\end{pmatrix}$$

$A$ va $B$ matritsalarning ko`paytmasi:
\begin{multline*}
	AB=\begin{pmatrix}
		a_{11}&a_{12}&a_{13}\\
		a_{21}&a_{22}&a_{23}\\
		a_{31}&a_{32}&a_{33}
	\end{pmatrix}\begin{pmatrix}
		b_{11}&b_{12}&b_{13}\\
	b_{21}&b_{22}&b_{23}\\
	b_{31}&b_{32}&b_{33}\\
\end{pmatrix}
=\begin{pmatrix}
	\sum\limits_{j=1}^{3}a_{1j}b_{j1}&\sum\limits_{j=1}^{3}a_{1j}b_{j2}&\sum\limits_{j=1}^{3}a_{1j}b_{j3}\\
	\sum\limits_{j=1}^{3}a_{2j}b_{j1}&\sum\limits_{j=1}^{3}a_{2j}b_{j2}&\sum\limits_{j=1}^{3}a_{2j}b_{j3}\\
	\sum\limits_{j=1}^{3}a_{3j}b_{j1}&\sum\limits_{j=1}^{3}a_{3j}b_{j2}&\sum\limits_{j=1}^{3}a_{3j}b_{j3}\\
\end{pmatrix}
\end{multline*}
ya'ni, ko`paytma-matritsaning $i$-satr va $k$-ustunda turgan elementi, $A$ matritsa $i$-satrining $B$ matritsa $k$-ustuni elementlari bilan ko`paytmalarining yig`indisiga teng.

Umumiy holda kki matritsa ko`paytmasi uchun o`rin almashtirish qoidasi bajarilmaydi, ya'ni $AB\ne BA$. 

Barcha elementlari nolga teng bo`lgan matritsa nol matritsa deyiladi:
$$\begin{pmatrix}
	0&0&0\\
	0&0&0\\
	0&0&0
\end{pmatrix}=0$$
Ushbu matritsaning ixtiyoriy $A$ matritsa bilan yig`indisi shu $A$ matritsaning o`ziga teng: $A+0=A$.

$$E=\begin{pmatrix}
	1&0&0\\
	0&1&0\\
	0&0&1
\end{pmatrix}$$ ko`rinishidagi matritsani birlik matritsa deyiladi.

Ushbu matritsani ixtiyoriy $A$ matritsaga ko`paytirilsa, shu matritsaning o`zi hosil bo`ladi: $EA=AE=A$. Birlik matritsada chiziqli almashtirish ayniyat ko`rinishiga keladi: $x=x^\prime$, $y=y^\prime$, $z=z^\prime$. 

Agar $A$ va $B$ matritsalarning ko`paytmasi uchun $AB=BA=E$ shart bajarilsa, $B$ matritsa $A$ matritsa uchun teskari matritsa dab ataladi.

$A$ matritsaga teskari bo`lgan matritsani $A^{-1}$ ko`rinishida ifodalanadi. Masalan, $B=A^{-1}$.

Har qanday aynimagan kvadrat matritsa $A$ ning teskari matritsasi mavjud. Teskari matritsa quyidagi formula orqali aniqlanadi:
$$A^{-1}\begin{pmatrix}
	A_{11}/D_A&	A_{21}/D_A&	A_{31}/D_A\\
	A_{21}/D_A&	A_{22}/D_A&	A_{23}/D_A\\
	A_{31}/D_A&	A_{32}/D_A&	A_{33}/D_A\\
\end{pmatrix}$$
bu yerda $A_{mn}$ -- matritsa $a_{mn}$ elementining algebraik to`ldiruvchisi. Ya'ni $A$ matritsa determinantidagi $m$-satr va $n$-ustunni o`chirib tashlashdan hosil bo`lgan ikkinchi tartibli minorning $(-1)^{m+n}$ ga ko`paytmasi.

$$X=\begin{pmatrix}
	x_1\\x_2\\x_3
\end{pmatrix}$$
ko`rinishidagi matritsalarni ustun-matritsa deb aytiladi. 

$AX$ ko`paytma quyidagi tenglikdan aniqlanadi:
$$AX=\begin{pmatrix}
	a_{11}&a_{12}&a_{13}\\
	a_{21}&a_{22}&a_{23}\\
	a_{31}&a_{32}&a_{33}
\end{pmatrix}\begin{pmatrix}
x_1\\x_2\\x_3
\end{pmatrix}=\begin{pmatrix}
a_{11}x_1+a_{12}x_2+a_{13}x_3\\
a_{21}x_1+a_{22}x_2+a_{23}x_3\\
a_{31}x_1+a_{32}x_2+a_{33}x_3\\
\end{pmatrix}$$

$$\begin{cases}
	a_{11}x_1+a_{12}x_2+a_{13}x_3=b_1\\
	a_{21}x_1+a_{22}x_2+a_{23}x_3=b_2\\
	a_{31}x_1+a_{32}x_2+a_{33}x_3=b_3\\
\end{cases}$$
ko`rinishidagi tenglamalar sistemasini $AX=B$ ko`rinishida ham yozish mumkin. Bu yerda
$$A=\begin{pmatrix}
		a_{11}&a_{12}&a_{13}\\
	a_{21}&a_{22}&a_{23}\\
	a_{31}&a_{32}&a_{33}
\end{pmatrix},\ \ X=\begin{pmatrix}
x_1\\x_2\\x_3
\end{pmatrix},\ \ B=\begin{pmatrix}
b_1\\b_2\\b_3
\end{pmatrix}$$

$$\begin{pmatrix}
	a_{11}&a_{12}&a_{13}\\
	a_{21}&a_{22}&a_{23}\\
	a_{31}&a_{32}&a_{33}
\end{pmatrix}$$
ko`rinishidagi matritsaning xarakteristik tenglamasi:
$$\begin{vmatrix}
	a_{11}-\lambda&a_{12}&a_{13}\\
	a_{21}&a_{22}-\lambda&a_{23}\\
	a_{31}&a_{32}&a_{33}-\lambda
\end{vmatrix}=0$$

Ushbu tenglamaning ildizlari $\lambda_1$, $\lambda_2$, $\lambda_3$ matritsaning xarakteristik sonlari deyiladi; agar boshlang`ich matritsa simmetrik bo`lsa, ushbu sonlar har doim haqiqiy bo`ladi.

$\lambda_1$, $\lambda_2$, $\lambda_3$ lardan bittasining qiymatiga teng $\lambda$ yechimga ega bo`lgan  hamda determinanti nolga teng bo`lgan quyidagi tenglamalar sistemasi, ushbu $\lambda$ songa tegishli bo`lgan $(\xi_1;\ \xi_2;\ \xi_3)$ sonlar uchligini aniqlaydi:

	$$
\begin{cases}(a_{11}-\lambda)\xi_1+a_{12}\xi_2+a_{13}\xi_3=0\\
	a_{21}\xi_1+(a_{22}-\lambda)\xi_2+a_{23}\xi_3=0\\
	a_{31}\xi_1+a_{32}\xi_2+(a_{33}-\lambda)\xi_3=0
\end{cases}
	$$

Ushbu sonlar uchligi $(\xi_1;\ \xi_2;\ \xi_3)$ matritsaning xususiy vektori deb ataladigan $\textbf{r}=\xi_1\textbf{i}+\xi_2\textbf{j}+\xi_3\textbf{k}$ vektorni aniqlaydi. 


\begin{enumerate}\setcounter{enumi}{394}
	
	\item $x=x^\prime+y^\prime+z^\prime$, $y=x^\prime+y^\prime$, $z=x^\prime$ chiziqli almashtirish formulalari hamda $x^\prime$, $y^\prime$, $z^\prime$ koordinata sistemasida nuqtalarning koordinatalari: $(1;-1;1)$, $(3;-2;-1)$, $(-1;-2;-3)$ berilgan. Ushbu nuqtalarning $x,\ y,\ z$ koordinatalar sistemasidagi koordinatalarini aniqlang.
	
	$\triangle$ Nuqtalarning koordinatalarini almashtirish formulalariga qo`yib, quyidagi tengliklarni hosil qilamiz: agar $x^\prime=1,\ y^\prime=1,\ z^\prime=1$ bo`lsa, $x=1,\ y=0,\ z=1$, ya'ni $(1;0;1)$; agar $x^\prime=3,\ y^\prime=-2,\ z^\prime=-1$ bo`lsa, $x=0,\ y=1,\ z=3$, ya'ni $(0;1;3)$; agar $x^\prime=-1,\ y^\prime=-2,\ z^\prime=-1$ bo`lsa, $x=-6,\ y=-3,\ z=-1$, ya'ni $(-6;-3;-1).\ \blacktriangle$
	
	\item Oldingi masalada $x$, $y$, $z$ koordinatalardan $x^\prime$, $y^\prime$, $z^\prime$ koordinatalarga o`tish uchun chiziqli almashtirish formulalarini yozing.
	
	$\triangle$ Uchinchi tenglikdan $x^\prime=x$ ekanligi ma'lum; ikkinchi tenglikdan uchinchi tenglikni ayiramiz: $y^\prime=y-z$; birinchi tenglikdan ikkinchi tenglikni ayiramiz: $z^\prime=x-y.\ \blacktriangle$ 
	
	\item $x=x^\prime+2y^\prime$, $y=3x^\prime+4y^\prime$ ko`rinishidagi chiziqli almashtirish formulalari berilgan. Qaysi nuqtalarda uning koordinatalari o`zgarmaydi?
	
	$\triangle$ $x=x^\prime$ va $y=y^\prime$ uchun $x$ va $y$ larni topish kerak. Ya'ni, $x=x+2y$, $y=3x+4y$. U holda, $x=x^\prime=0$, $y=y^\prime=0.\ \blacktriangle$
	
	\item Qaysi nuqtalarda $x=3x^\prime-2y^\prime$, $y=5x^\prime-4y^\prime$ chiziqli almashtirish koordinatalari o`zgarmaydi?
	
	$\triangle$ Masala shartiga binoan, $x=3x-2y$, $y=5x-4y$. Ya'ni chiziqli almashtirish bir xil koordinatali  $(t;t)$ nuqtalarda o`zgarmaydi. $\blacktriangle$
	
	\item Matritsalar yig`indisini hisoblang:
	$$A=\begin{pmatrix}
		3&5&7\\
		2&-1&0\\
		4&3&2
	\end{pmatrix},\ \ B=\begin{pmatrix}
	1&2&4\\
	2&3&-2\\
	-1&0&1
\end{pmatrix}$$

$$\triangle\ A+B=\begin{pmatrix}
	3+1&5+2&7+4\\
	2+2&-1+3&0-2\\
	4-1&3+0&2+1
	
\end{pmatrix}=\begin{pmatrix}
4&7&11\\
4&2&-2\\
3&3&3
\end{pmatrix}.\ \blacktriangle$$

\item $$A=\begin{pmatrix}
	3&5\\
	4&1
\end{pmatrix},\ \ B=\begin{pmatrix}
2&3\\
1&-2
\end{pmatrix}$$ bo`lsa, $2A+5B$ ni hisoblang.

$\triangle$
$$2A=\begin{pmatrix}
	6&10\\
	8&2
\end{pmatrix},\ \ 5B=\begin{pmatrix}
10&15\\
5&-10
\end{pmatrix},\ \ 2A+5B=\begin{pmatrix}
16&25\\
13&-8
\end{pmatrix}.\ \blacktriangle$$

\item Agar 
$$A=\begin{pmatrix}
	1&3&1\\
	2&0&4\\
	1&2&3
	\end{pmatrix}, \mbox{va}\ \ B=\begin{pmatrix}
	2&1&0\\
	1&-1&2\\
	3&2&1
\end{pmatrix}$$ bo`lsa, $AB$ va $BA$ ni hisoblang.

$$\triangle\ AB=\begin{pmatrix}
	1\cdot2+3\cdot1+1\cdot3&1\cdot1+3(-1)+1\cdot2&1\cdot0+3\cdot2+1\cdot1\\
	2\cdot2+0\cdot1+4\cdot3&2\cdot1+0(-1)+4\cdot2&2\cdot0+0\cdot2+4\cdot1\\
	1\cdot2+2\cdot1+3\cdot3&1\cdot1+2(-1)+3\cdot2&1\cdot0+2\cdot2+3\cdot1
\end{pmatrix}=\begin{pmatrix}
8&0&7\\
16&10&4\\
13&5&7
\end{pmatrix}$$

$$BA=\begin{pmatrix}
	2\cdot1+1\cdot2+0\cdot1& 2\cdot3+1\cdot0+0\cdot2&2\cdot1+1\cdot4+0\cdot3\\
	1\cdot1-1\cdot2+2\cdot1& 1\cdot3-1\cdot0+2\cdot2&1\cdot1-1\cdot4+2\cdot3\\
	3\cdot1+2\cdot2+1\cdot1& 3\cdot3+2\cdot0+1\cdot2& 3\cdot1+2\cdot4+1\cdot3
\end{pmatrix}=\begin{pmatrix}
4&6&6\\
1&7&3\\
8&11&14
\end{pmatrix}.\ \blacktriangle$$

\item Agar $A=\begin{pmatrix}
	3&2\\1&4\end{pmatrix}$ bo`lsa, $A^3$ ni toping.

$$\triangle\ A^2=\begin{pmatrix}
	3&2\\1&4\end{pmatrix}\begin{pmatrix}
	3&2\\1&4
\end{pmatrix}=\begin{pmatrix}
9+2&6+8\\
3+4&2+16
\end{pmatrix}=\begin{pmatrix}
11&14\\
7&18
\end{pmatrix}$$
$$A^3=A^2\cdot A\begin{pmatrix}
	11&14\\7&18
\end{pmatrix}\begin{pmatrix}
3&2\\1&4
\end{pmatrix}=\begin{pmatrix}
33+14&22+56\\21+18&14+72
\end{pmatrix}=\begin{pmatrix}
47&78\\39&86
\end{pmatrix}.\ \blacktriangle$$

\item Matritsa - ko`phadning qiymatini hisoblang: $2A^2+3A+5E$. Bu yerda $A=\begin{pmatrix}
	1&1&2\\1&3&1\\4&1&1
\end{pmatrix}$, $E$ - uchinchi tartibli birlik matritsa.

$$\triangle A^2=\begin{pmatrix}
	1&1&2\\1&3&1\\4&1&1
\end{pmatrix}\begin{pmatrix}
1&1&2\\1&3&1\\4&1&1
\end{pmatrix}=\begin{pmatrix}
10&6&5\\8&11&6\\9&8&10
\end{pmatrix},\ \ 2A^2=\begin{pmatrix}
20&12&10\\16&22&12\\18&16&20
\end{pmatrix},$$
$$3A=\begin{pmatrix}
	3&3&6\\3&9&3\\12&3&3
\end{pmatrix},\ \ 5E=5\begin{pmatrix}
1&0&0\\0&1&0\\0&0&1
\end{pmatrix}=\begin{pmatrix}
5&0&0\\0&5&0\\0&0&5
\end{pmatrix},$$
$$2A^2+3A+5E=\begin{pmatrix}
	28&15&16\\19&36&15\\30&19&28
\end{pmatrix}.\ \blacktriangle$$

\item Ikkita chiziqli almashtirish formulalari berilgan: $x=a_{11}x^\prime+a_{12}y^\prime$, $y=a_{21}x^\prime+a_{22}y^\prime$ va $x^\prime=b_{11}x^{\prime\prime}+b_{12}y^{\prime\prime}$, $y^\prime=b_{21}x^{\prime\prime}+b_{22}y^{\prime\prime}$. Ikkinchi almashtirish formulasidan $x^\prime$ va $y^\prime$ lar-ning qiymatlarini birinchi formulaga qo`yib, $x$ va $y$ ni $x^{\prime\prime}$ hamda $y^{\prime\prime}$ orqali ifodalash mumkin bo`ladi. Ushbu hosil bo`lgan chiziqli almashtirish matritsasining birinchi va ikkinchi almashtirishlar matritsalari ko`paytmasiga teng ekanligini isbotlang.

$\triangle$
Masala shartidan ma'lumki,
$$x=a_{11}(b_{11}x^{\prime\prime}+b_{12}y^{\prime\prime})+a_{12}(b_{21}x^{\prime\prime}+b_{22}y^{\prime\prime})=(a_{11}b_{11}+a_{12}b_{12})x^{\prime\prime}+(a_{11}b_{12}+a_{12}b_{22})y^{\prime\prime}$$
$$y=a_{21}(b_{11}x^{\prime\prime}+b_{12}y^{\prime\prime})+a_{22}(b_{21}x^{\prime\prime}+b_{22}y^{\prime\prime})=(a_{21}b_{11}+a_{22}b_{21})x^{\prime\prime}+(a_{21}b_{12}+a_{22}b_{22})y^{\prime\prime}$$

Hosil bo`lgan chiziqli almashtirish matritsasi quyidagi ko`rinishda bo`ladi:
$$\begin{pmatrix}
	a_{11}b_{11}+a_{12}b_{21}&a_{11}b_{12}+a_{12}b_{22}\\
	a_{21}b_{11}+a_{22}b_{21}&a_{21}b_{12}+a_{22}b_{22}\\
\end{pmatrix}$$
Ya'ni ushbu matritsa $\begin{pmatrix}
	a_{11}&a_{12}\\a_{21}&a_{22}
\end{pmatrix}$ hamda $\begin{pmatrix}
b_{11}&b_{12}\\b_{21}&b_{22}
\end{pmatrix}$ matritsalarning ko`paytmasidir. $\blacktriangle$
\item $A=\begin{pmatrix}
	3&2&2\\1&3&1\\5&3&4
\end{pmatrix}$ matritsaga teskari bo`lgan matritsani toping.

$\triangle$ $A$ matritsaning determinantini hisoblaymiz:
$$D_A=\begin{vmatrix}
	3&2&2\\1&3&1\\5&3&4
\end{vmatrix}=27+2-24=5.$$
	Ushbu determinant elementlarining algebraik to`ldiruvchilarini aniqlaymiz:
	$$A_{11}=\begin{vmatrix}
		3&1\\3&4
	\end{vmatrix}=9,\ \ A_{21}=-\begin{vmatrix}
	2&2\\3&4
\end{vmatrix}=-2,\ \ A_{31}=\begin{vmatrix}
2&2\\3&1
\end{vmatrix}=-4$$
$$A_{12}=-\begin{vmatrix}
	1&1\\5&4
\end{vmatrix}=1,\ \ A_{22}=\begin{vmatrix}
3&2\\5&4
\end{vmatrix}=2,\ \ A_{32}=-\begin{vmatrix}
3&2\\1&1
\end{vmatrix}=-1,$$
$$A_{13}=\begin{vmatrix}
	1&3\\5&3
\end{vmatrix}=-12,\ \ A_{23}=-\begin{vmatrix}
3&2\\5&3\end{vmatrix}=1,\ \ A_{33}=\begin{vmatrix}
3&2\\1&3
\end{vmatrix}=7.
$$
Shunday qilib,
$$A^{-1}=\begin{pmatrix}
	9/5&-2/5&-4/5\\
	1/5&2/5&-1/5\\
	-12/5&1/5&7/5
\end{pmatrix}.\ \blacktriangle
	$$
	
	\item Tenglamalar sistemasini yeching:
	$$\begin{cases}
		2x+3y+2z=9\\
		x+2y-3z=14\\
		3x+4y+z=16
	\end{cases}$$

$\triangle$ Ushbu tenglamalar sistemasini $AX=B$ ko`rinishida qayta yozib olamiz. Bu yerda
$$A=\begin{pmatrix}
	2&3&2\\
	1&2&-3\\
	3&4&1
\end{pmatrix},\ \ X=\begin{pmatrix}
x\\y\\
z
\end{pmatrix},\ \ B=\begin{pmatrix}
9\\14\\16
\end{pmatrix}$$
Ushbu  tenglamaning yechimi $X=A^{-1}B$ ko`rinishda bo`ladi. Demak, $A^{-1}$ ni hisoblashimiz zarur. Buning uchun eng birinchi navbatda:
$$D_A=\begin{vmatrix}
		2&3&2\\
	1&2&-3\\
	3&4&1
\end{vmatrix}=28-30-4=-5.$$
Endi esa determinantning algebraik to`ldiruvchilarini aniqlaymiz:
$$A_{11}=\begin{vmatrix}
	2&-3\\4&1
\end{vmatrix}=14,\ \ A_{21}=-\begin{vmatrix}
3&2\\4&1
\end{vmatrix}=5,\ \ A_{31}=\begin{vmatrix}
3&2\\2&-3
\end{vmatrix}=-13$$

$$A_{12}=-\begin{vmatrix}
	1&-3\\3&1\end{vmatrix}=-10,\ \ A_{22}=\begin{vmatrix}
	2&2\\3&1
\end{vmatrix}=-4,\ \ A_{32}=-\begin{vmatrix}
2&2\\1&-3
\end{vmatrix}=8,
	$$
	$$A_{13}=\begin{vmatrix}
		1&2\\3&4
	\end{vmatrix}=-2,\ \ A_{23}=-\begin{vmatrix}
		2&3\\3&4\end{vmatrix}=1,\ \ A_{33}=\begin{vmatrix}
		2&3\\1&2
	\end{vmatrix}=1.$$
Shunday qilib, 
$$A^{-1}=-\frac{1}{6}\begin{pmatrix}
	14&5&-13\\-10&-4&8\\-2&1&1
\end{pmatrix}$$
Bu yerdan 
$$X=-\frac{1}{6}\begin{pmatrix}
	14&5&-13\\-10&-4&8\\-2&1&1
\end{pmatrix}\begin{pmatrix}
9&14&16
\end{pmatrix}=-\frac{1}{6}\begin{pmatrix}
126+70-208\\
-90-56+128\\
18+14+16
\end{pmatrix}=-\frac{1}{6}\begin{pmatrix}
-12\\-18\\12
\end{pmatrix}=\begin{pmatrix}
2\\3\\-2
\end{pmatrix}
$$
Demak, $x=2$, $y=3$, $z=-2.\ \blacktriangle$

\item 
$\begin{pmatrix}
	5&2\\4&3
\end{pmatrix}$ ko`rinishidagi matritsa berilgan. Uning xarakteristik vektorlari va xarakteristik sonlarini toping.

$\triangle$ Xarakteristik tenglamani tuzamiz:
$$\begin{vmatrix}
	5-\lambda&2\\4&3-\lambda
\end{vmatrix}=0,\ \mbox{yoki}\ \ (5-\lambda)(3-\lambda)-8=0,\ \mbox{ya'ni}\ \ \lambda^2-8\lambda+7=0.$$

Xarakteristik sonlar: $\lambda_1=1$ va $\lambda_2=7$. Birinchi xarakteristik songa mos keladigan xususiy vektorni quyidagi tenglamalar sistemasidan aniqlaymiz:
$$\begin{cases}
	(5-\lambda_1)\xi_{1}^{\prime}+2\xi_2^\prime=0\\
	4\xi_1^\prime+(3-\lambda_1)\xi_2^\prime=0
\end{cases}$$
$\lambda_1=1$ bo`lgani uchun $\xi_1^\prime$ va $\xi_2^\prime$ lar o`zaro $2\xi_1^\prime+\xi_{2}^{\prime}$ ko`rinishida bog`langan. $\xi_1^\prime=\alpha$ ($\alpha\ne0$ -  ixtiyoriy son) deb hisoblab, $\xi_2^\prime=-2\alpha$ ni hosil qilamiz. U holda $\lambda_1$ xarakteristik songa mos keluvchi xarakteristik vektor tenglamasi: $\textbf{r}_1=\alpha\textbf{i}-2\alpha\textbf{j}$  bo`ladi.

Endi ikkinchi xususiy vektorni aniqlaymiz.

$$\begin{cases}
	(5-\lambda_2)\xi_{1}^{\prime\prime}+2\xi_2^{\prime\prime}=0\\
	4\xi_1^{\prime\prime}+(3-\lambda_2)\xi_2^{\prime\prime}=0
\end{cases}$$
$\lambda_2=7$ qiymatni tenglamalar sistemasiga qo`yib, $\xi_{1}^{\prime\prime}-\xi_{2}^{\prime\prime}=0$ munosabatni hosil qilamiz. Ya'ni, $\xi_{1}^{\prime\prime}=\xi_{2}^{\prime\prime}=\beta\ne0$. U holda ikkinchi xarakteristik songa mos keladigan xususiy vektor: $\textbf{r}_2=\beta\textbf{i}+\beta\textbf{j}$ tenglama bilan ifodalanadi. $\blacktriangle$


\item $\begin{pmatrix}
	5&-171\\-1&5&-1\\1&-1&3
\end{pmatrix}$ ko`rinishidagi matritsa berilgan. Uning xarakteristik sonlari va xarakteristik vektorlarini toping.

$\triangle$ Xarakteristik tenglamani tuzamiz:
$$\begin{vmatrix}
	3-\lambda&-1&1\\
	-1&5-\lambda&-1\\
	1&-1&3-\lambda
\end{vmatrix}=0$$
yoki 
$$(3-\lambda)\left[(5-\lambda)(3-\lambda)-1\right]+(-3+\lambda+1)+(1-5+\lambda)=0$$
Elementar almashtirishlardan so`ng tenglama $(3-\lambda)(\lambda^2-8\lambda+12)=0$ ko`rinishga keladi. Bu yerdan $\lambda_1=2$, $\lambda_2=3$, $\lambda_3=6$.

\item Ikkita chiziqli almashtirish formulalari berilgan: $x=a_{11}x^\prime+a_{12}y^\prime+a_{13}z^\prime$, $y=a_{21}x^\prime+a_{22}y^\prime+a_{23}z^\prime$,
$z=a_{31}x^\prime+a_{32}y^\prime+a_{33}z^\prime$
va 
$x^\prime=b_{11}x^{\prime\prime}+b_{12}y^{\prime\prime}+b_{13}z^{\prime\prime}$,
$y^\prime=b_{21}x^{\prime\prime}+b_{22}y^{\prime\prime}+b_{23}z^{\prime\prime}$,
$z^\prime=b_{31}x^{\prime\prime}+b_{32}y^{\prime\prime}+b_{33}z^{\prime\prime}$
. Ikkinchi almashtirish formulasidan $x^\prime$ va $y^\prime$ larning qiymatlarini birinchi formulaga qo`yib, $x$ va $y$ ni $x^{\prime\prime}$ hamda $y^{\prime\prime}$ orqali ifodalash mumkin bo`ladi. Ushbu hosil bo`lgan chiziqli almashtirish matritsasining birinchi va ikkinchi almashtirishlar matritsalari ko`paytmasiga teng ekanligini isbotlang.

\item $x=6x'+y'-2z'$, $y=-18x'+2y'+6z'$, $z=2x'+2y'$ almashtirish berilgan. Ushbu almashtirish natijasida qaysi nuqtalar koordinatalarining qiymati ikki baravar ortadi?

\item Ikkita chiziqli almashtirish berilgan: $x=x'+y'+2z'$, $y=x'+2y'+6z'$, $z=2x'+3y'$; $x=2x'+2z'$, $y=x'+3y'+4z'$, $z=x'+3y'+2z'$. Ushbu ikkala almashtirish ham bir xil natija beradigan nuqtalarni toping.

\item $x=x'\cos\alpha-y'\sin\alpha$, $y=x'\sin\alpha+y'\cos\alpha$ almashtirish bajarilganda, koordinatalari o`zgarmay qoladigan nuqtalarni aniqlang.

\item $x=x'\cos\alpha-y'\sin\alpha$, $y=x'\sin\alpha+y'\cos\alpha$ almashtirish bajarilganda koordinatalari o`zaro almashadigan nuqtalarni aniqlang.

\item $A=\begin{pmatrix}
	5&8&4\\3&2&5\\7&6&0
\end{pmatrix}$ matritsa berilgan. Birlik matritsa hosil bo`lishi uchun $A$ matritsaga qanday $B$ matritsani qo`shish kerak?

\item $A=\begin{pmatrix}
	2&1&1\\1&2&1\\1&1&2
\end{pmatrix}$ matritsa berilgan. $A^2+A+E$ ni hisoblang.

\item $A=\begin{pmatrix}
	10&20&-30\\0&10&20\\0&0&10
\end{pmatrix}$ matritsaga teskari matritsani toping.

\item Quyidagi tenglamalar sistemasini matritsa yordamida yeching:
$$\begin{cases}
	3x+4y=11\\5y+6z=28\\x+2z=7
\end{cases}$$

\item $\begin{pmatrix}
	7&4\\5&6
\end{pmatrix}
$ matritsa uchun xarakteristik sonlar va normallashgan xususiy vektorlarni hisoblang.

\item $\begin{pmatrix}
	1&1&3\\1&5&1\\3&1&1
\end{pmatrix}$ matritsa uchun xarakteristik sonlar va normallashgan xususiy vektorlarni hisoblang.
\end{enumerate}

\section{2-tartibli egri chiziq va sirtlarning umumiy tenglamalarini kanonik ko`rinishga keltirish}
\section{Matritsa rangi. Ekvivalent matritsalar}


\begin{enumerate}
	\setcounter{enumi}{427}
	
	\item Matritsa rangini aniqlang: $\begin{pmatrix}
		1&2&3&4\\2&4&6&8\\3&6&9&12
	\end{pmatrix}$
	
	$\triangle$ Ushbu matritsaning barcha ikkinchi va uchinchi tartibli minorlari nolga teng, chunki ushbu minorlarning satr elementlari o`zaro ekvivalent. Birinchi tartibli minorlar (matritsa elementlari) noldan farqli. Demak matritsa rangi 1 ga teng. $\blacktriangle$
	
	\item Matritsa rangini aniqlang: $\begin{pmatrix}
		1&0&0&0&5\\0&0&0&0&0\\2&0&0&0&11\\
	\end{pmatrix}$

$\triangle$ Ushbu matritsaning 2-satrini o`chirib tashlaymiz. Keyin esa 2, 3 va 4-ustunlarni ham o`chirib tashlaymiz. Natijada $\begin{pmatrix} 1&5\\2&11\end{pmatrix}$ ko`rinishidagi matritsa hosil bo`ladi. Ushbu matritsa masala shartida berilgan matritsa bilan ekvivalent. $\begin{vmatrix}
	1&5\\2&11
\end{vmatrix}=1\ne0$ bo`lgani uchun, berilgan matritsaning rangi 2 ga teng. $\blacktriangle$

\item Matritsa rangini aniqlang: $A=\begin{pmatrix}
	3&5&7\\1&2&3\\1&3&5
\end{pmatrix}$.

$\triangle$ 1- va 3-satr elementlarini qo`shamiz, so`ng 1-satrning 4-elementiga bo`lamiz:
$$A=\begin{pmatrix}
	3&5&7\\1&2&3\\1&3&5
\end{pmatrix}\sim\begin{pmatrix}
4&8&12\\1&2&3\\1&3&5
\end{pmatrix}\sim\begin{pmatrix}
1&2&3\\1&2&3\\1&3&5
\end{pmatrix}
$$

1-satr elementlaridan 2-satr elementlarini ayiramiz va 1-satrni o`chirib tashlaymiz:
$$\begin{pmatrix}
	1&2&3\\1&2&3\\1&3&5
\end{pmatrix}\sim\begin{pmatrix}
0&0&0\\1&2&3\\1&3&5
\end{pmatrix}\sim\begin{pmatrix}
1&2&3\\1&3&5
\end{pmatrix}$$
$\begin{vmatrix}
	1&2\\1&3
\end{vmatrix}\ne0$ bo`lgani uchun  matritsaning rangi 2 ga teng. $\blacktriangle$

\item Matritsa rangini aniqlang: $A=\begin{pmatrix}
	4&3&2&2\\0&2&1&1\\0&0&3&3
\end{pmatrix}$.

$\triangle$ 4-ustun elementlaridan 3-ustun elementlarini ayiramiz va 4-ustunni o`chirib tashlaymiz:
$$A=\begin{pmatrix}
	4&3&2&2\\0&2&1&1\\0&0&3&3
\end{pmatrix}\sim\begin{pmatrix}
4&3&2&0\\0&2&1&0\\0&0&3&0
\end{pmatrix}\sim\begin{pmatrix}
4&3&2\\0&2&1\\0&0&3
\end{pmatrix}$$
 $\begin{vmatrix}
	4&3&2\\0&2&1\\0&0&3
\end{vmatrix}=24\ne0$. Demak matritsa rangi 3 ga teng. $\blacktriangle$

\item Matritsa rangini  va bazis minorlarini aniqlang:
$A=\begin{pmatrix}
	1&0&2&0&0\\0&1&0&2&0\\2&0&4&0&
\end{pmatrix}$

\begin{multline*}\triangle\ \begin{pmatrix}
	1&0&2&0&0\\0&1&0&2&0\\2&0&4&0&
\end{pmatrix}\sim\begin{pmatrix}
1&0&2&0\\0&1&0&2\\2&0&4&0
\end{pmatrix}\sim\begin{pmatrix}
1&0&2&0\\0&1&0&2\\1&0&2&0
\end{pmatrix}\sim\begin{pmatrix}
1&0&2&0\\0&1&0&2\\0&0&0&0
\end{pmatrix}\sim\\
\sim\begin{pmatrix}
1&0&2&0\\0&1&0&2
\end{pmatrix};\ r(A)=2.
\end{multline*} 
Matritsaning noldan farqli bo`lgan ikkinchi tartibli minorlariga bazis minorlar deyiladi:
$$\begin{vmatrix}
	1&0\\0&1
\end{vmatrix},\ \ \begin{vmatrix}
1&0\\0&2
\end{vmatrix},\ \ \begin{vmatrix}
0&2\\1&0
\end{vmatrix},\ \ \begin{vmatrix}
2&0\\0&2
\end{vmatrix},\ \ \begin{vmatrix}
0&1\\2&0
\end{vmatrix},\ \ \begin{vmatrix}
0&2\\2&0
\end{vmatrix},\ \ \begin{vmatrix}
1&0\\0&4
\end{vmatrix},\ \ \begin{vmatrix}
0&2\\4&0
\end{vmatrix}
$$
Shunday qilib $A$ matritsa 8 ta bazis minorlarga ega ekan. $\blacktriangle$

\item $A$ matritsada nechta ikkinchi tartibli minor mavjudligini aniqlang:
$A=\begin{pmatrix}
	a_{11}&a_{12}&a_{13}\\
	a_{21}&a_{22}&a_{23}\\
	a_{31}&a_{32}&a_{33}
\end{pmatrix}$.

$\triangle$ Matritsada $C_3^2\cdot C_3^2=3\cdot3=9$ ta ikkinchi tartibli minor mavjud:
$$\begin{vmatrix}
	a_{22}&a_{23}\\
	a_{32}&a_{33}
\end{vmatrix},\ \ \begin{vmatrix}
a_{21}&a_{23}\\
a_{31}&a_{33}
\end{vmatrix},\ \ \begin{vmatrix}
a_{21}&a_{22}\\
a_{31}&a_{32}
\end{vmatrix},\ \ \begin{vmatrix}
a_{12}&a_{13}\\
a_{32}&a_{33}
\end{vmatrix},$$
$$\begin{vmatrix}
	a_{11}&a_{13}\\
	a_{31}&a_{33}
\end{vmatrix},\ \ \begin{vmatrix}
a_{11}&a_{12}\\
a_{21}&a_{32}
\end{vmatrix},\ \ \begin{vmatrix}
a_{12}&a_{13}\\
a_{22}&a_{23}
\end{vmatrix},\ \ \begin{vmatrix}
a_{11}&a_{13}\\
a_{21}&a_{23}
\end{vmatrix},\ \ \begin{vmatrix}
a_{11}&a_{12}\\
a_{21}&a_{22}
\end{vmatrix}.\ \blacktriangle$$

\item Matritsa rangini aniqlang: $A=\begin{pmatrix}
	\lambda&5\lambda&-\lambda\\
	2\lambda&\lambda&10\lambda\\
	-\lambda&-2\lambda&-3\lambda
\end{pmatrix}$

\item Matritsa rangini aniqlang: $A=\begin{pmatrix}
	1&2&3&6\\2&3&1&6\\3&1&2&6
	\end{pmatrix}$

\item Matritsa rangini va bazis minorlarini aniqlang: $A=\begin{pmatrix}
	0&2&0&0\\1&0&0&4\\0&0&3&0
\end{pmatrix}$

\item Matritsa rangini va bazis minorlarini aniqlang: $A=\begin{pmatrix}
	1&2&1&3&4\\3&4&2&6&8\\1&2&1&3&4
\end{pmatrix}$
\end{enumerate}


\section{n ta noma'lumli m ta chiziqli tenglamalar sistemasini yechish}

Tenglamalar sistemasini yeching:
\begin{enumerate}\setcounter{enumi}{440}
	\item $\begin{cases}3x_1+2x_2=4\\
		x_1-4x_2=-1\\
		7x_1+10x_2=12\\
		5x_1+6x_2=8\\
		3x_1-16x_2=8
	\end{cases}$

\item $\begin{cases}
	x_1+5x_2+4x_3=1\\
	2x_1+10x_2+8x_3=3\\
	3x_1+15x_2+12x_3=5
\end{cases}$

\item $\begin{cases}
	x_1-3x_2+2x_3=-1\\
	x_1+9x_2+6x_3=3\\
	x_1+3x_2+4x_3=1
\end{cases}$
\end{enumerate}

\section{Chiziqli tenglamalar sistemasini Gauss usulida yechish}


\begin{enumerate}\setcounter{enumi}{443}
	\item Tenglamalar sistemasini yeching:
	$$\begin{cases}
		36,47x+5,28y+6,34z=12,26&\ (a)\\
		7,33x+28,74y+5,86z=15,15&\ (b)\\
		4,63x+6,31y+26,17z=25,22&\ (d)
	\end{cases}$$
$\triangle$ (a) tenglamani 36,47 ga bo`lamiz va quyidagini hosil qilamiz:
$$x+0,1447y+0,1738z=0,3361\ \ \ (*)$$

(*) tenglamani 7,33 ga ko`paytiramiz va natijani (b) dan ayiramiz:
$$27,6793y+4,586z=12,6864$$ 
So`ngra (*) tenglamani 4,63 ga ko`paytiramiz va natijani (d) dan ayiramiz:
$$5,64y+25,3653z=23,6639$$

Shunday qilib, tenglamalar sistemasi quyidagi ko`rinishga keladi:
$$\begin{cases}
	27,6793y+4,586z=12,6864&\ (e)\\
	5,64y+25,3653z=23,6639&\ (f)\\
\end{cases}$$
(e) tenglamani 27,68 ga bo`lamiz:
$$y+0,1657z=0,4583\ \ \ (**)$$
(**) tenglamani 5,64 ga ko`paytirib, natijani (f) dan ayiramiz va $24,4308z=21,0791$ hosil bo`ladi. Bundan $z=0,8628$. U holda 
$$y=0,4583-0,1657\cdot0,8628=0,3153$$
$$x=0,3361-0,1447\cdot0,3153-0,1738\cdot0,8628=0,1405$$
Tenglamalar sistemasining ildizlari: $x=0,1405$, $y=0,3153$ va $z=0,8628$. 

Amalda tenglamalar sistemasining o`zini emas, balki uning koeffitsiyentlari hamda ozod hadlaridan tuzilgan matritsani zinapoyasimon ko`rinishga keltirish qulayroq:
$$\begin{pmatrix}
36,47&5,28&6,34|&12,26\\
7,33&28,74&5,86|&15,15\\
4,63&6,31&26,17|&25,22
\end{pmatrix}
$$
\end{enumerate}
Tenglamalar sistemasini yeching:
\begin{enumerate}\setcounter{enumi}{445}
	\item $\begin{cases}
		2x_{1}+x_{2}-x_{3}=5\\
		x_{1}-2x_{2}+2x_{3}=-5\\
		7x_{1}+x_{2}-x_{3}=10
	\end{cases}$
\inlineitem $\begin{cases}
	x_{1}+x_{2}-x_{3}+x_{4}=4\\
	2x_{1}-x_{2}+3x_{3}-2x_{1}=1\\
	x_{1}-x_{3}+2x_{4}=6\\
	3x_{1}-x_{2}+x_{3}-x_{4}=0
\end{cases}$

\item $\begin{cases}
	3x_{1}-x_{2}+x_{3}+2x_{5}=18\\
	2x_{1}-5x_{2}+x_{4}+x_{5}=-7\\
	x_{1}-x_{4}+2x_{5}=8\\
	2x_{2}+x_{3}+x_{4}-x_{5}=10\\
	x_{1}+x_{2}-3x_{3}+x_{4}=1
\end{cases}$
\inlineitem $\begin{cases}
	4x_{1}+2x_{2}+3x_{3}=-2\\
	2x_{1}+8x_{2}-x_{3}=8\\
	9x_{1}+x_{2}+8x_{3}=0\\
\end{cases}$

\item $\begin{cases}
	0,04x-0,08y+4z=20\\
	4x+0,24y_0,08z=8\\
	0,09x+3y-0,15z=9
\end{cases}$
\inlineitem $\begin{cases}
	3,21x+0,71y+0,34z=6,12\\
	0,43x+4,11y+0,22z=5,71\\
	0,17x+0,16y+4,73z=7,06
\end{cases}$
\end{enumerate}

\section{Chiziqli tenglamalar sistemasini Jordan-Gauss usulida yechish}






\begin{enumerate} \setcounter{enumi}{451}
	\item Chiziqli tenglamalar sistemasining matritsasi quyidagi ko`rinishga ega:
	$$A=\begin{pmatrix}
	5&4&6&-1&7\\
	8&1&3&2&0\\
	0&1&5&3&-1\\
	7&-6&5&-4&3\\	
	\end{pmatrix}$$
	Ushbu tenglamalar sistemasini Jordan-Gauss usulida yechish paytida ruxsat etilgan element sifatida $a_{23}=3$ element tanlandi. Transponirlangan matritsaning $a_{24}^{\prime}$, $a_{13}^{\prime}$, $a_{44}^{\prime}$ elementlarini toping.
	
	$\triangle$ $a_{24}$ - ruxsat etilgan satr elementi bo`lgani uchun $a_{24}^{\prime}=a_{24}$ bo`ladi. $a_{13}$ element ruxsat etilgan  ustunga tegishli, demak $a_{13}^{\prime}=0$. $a_{44}^{\prime}$ elementni  to`rtburchak qoidasidan foydalanib aniqlaymiz:
	$$A=\begin{pmatrix}
		5&4&6&-1&7\\
		8&1&\boxed{3}\ldots&2&0\\
		0&1&\vdots5\ \ \ \ \ \vdots&3&-1\\
		7&-6&\vdots5\ldots\vdots&-4&3\\
	\end{pmatrix}$$
	
	$$a_{44}^{\prime}=a_{44}-\frac{a_{24}a_{43}}{a_{23}}=-4-\frac{2\cdot5}{3}=-7\frac{1}{3}.\ \blacktriangle$$
	\item Tenglamalar sistemasini yeching:
	$$\begin{cases}
		x_{1}+x_{2}-3x_{3}+2x_{4}=6\\
		x_{1}-2x_{2}-x_{4}=-6\\
		x_{2}+x_{3}+3x_{4}=16\\
		2x_{1}-3x_{2}+2x_{3}=6
	\end{cases}$$
	$\triangle$ Quyidagi jadvalga tenglama koeffitsiyentlari, ozod hadlar, koeffitsiyentlar va ozod hadlar yig`indisi ($\Sigma$- nazorat ustuni) ni yozamiz:
	\begin{center}
		\begin{tabular}{|c|c|c|c|c|c|}
			\hline
			\textbf{x}$_{1}$&\textbf{x}$_{2}$&\textbf{x}$_{3}$&\textbf{x}$_{4}$&\textbf{b}&$\Sigma$\\ \hline
			$\boxed{1}$&1&-3&2&6&7\\ \hline
			1&-2&0&-1&-6&-8\\ \hline
			0&1&1&3&16&21\\ \hline
			2&-3&2&0&6&7\\ \hline
		\end{tabular}
	\end{center}
Ruxsat etilgan element sifatida 1-tenglamadagi $x_{1}$ ning oldida turgan koeffitsiyentni tanladik. Ushbu element joylashgan satrni o`zgarishsiz yozamiz. 1-ustundagi hamma elementlarni (ruxsat etilgan elementdan tashqari) nollar bilan almashtiramiz. To`rtburchak usulini qo`llagan holda, jadvalning qolgan kataklarini to`ldiramiz (ushbu qoida $\Sigma$ ustun uchun ham o`rinli):
\begin{center}
	\begin{tabular}{|c|c|c|c|c|c|}
		\hline
		\textbf{x}$_{1}$&\textbf{x}$_{2}$&\textbf{x}$_{3}$&\textbf{x}$_{4}$&\textbf{b}&$\Sigma$\\ \hline
		1&1&-3&2&6&7\\ \hline
		0&-3&3&-3&-12&-15\\ \hline
		0&1&1&3&16&21\\ \hline
		0&-5&8&-4&-6&-7\\ \hline
	\end{tabular}
\end{center}
Nazorat ustunida mos satr elementlarining yig`indisi yozilgan. 2-satr elementlarini $-3$ ga bo`lamiz va quyidagi jadval hosil bo`ladi:
\begin{center}
	\begin{tabular}{|c|c|c|c|c|c|}
		\hline
		\textbf{x}$_{1}$&\textbf{x}$_{2}$&\textbf{x}$_{3}$&\textbf{x}$_{4}$&\textbf{b}&$\Sigma$\\ \hline
		1&1&-3&2&6&7\\ \hline
		0&$\boxed{1}$&-1&1&4&5\\ \hline
		0&1&1&3&16&21\\ \hline
		0&-5&8&-4&-6&-7\\ \hline
	\end{tabular}
\end{center}

2-satrning 2-elementini ruxsat etilgan element sifatida tanlaymiz. 1-ustunni o`zgarishsiz qayta yozamiz, 2-ustunning barcha elementlarini (ruxsat etilgan elementdan tashqari) nollar bilan almashtiramiz. 2-satrni o`zgarishsiz yozib chiqamiz, jadvalning qolgan elementlarini  to`rtburchak usulini qo`llagan holda to`ldirib chiqamiz:
\begin{center}
	\begin{tabular}{|c|c|c|c|c|c|}
		\hline
		\textbf{x}$_{1}$&\textbf{x}$_{2}$&\textbf{x}$_{3}$&\textbf{x}$_{4}$&\textbf{b}&$\Sigma$\\ \hline
		
		1&0&-2&1&1&1\\ \hline
		0&1&-1&1&4&5\\ \hline
		0&0&2&2&12&16\\ \hline
		0&0&3&1&14&18\\ \hline
	\end{tabular}
\end{center}
3-satr elementlarini 2 ga bo`lamiz:
\begin{center}
	\begin{tabular}{|c|c|c|c|c|c|}
		\hline
		\textbf{x}$_{1}$&\textbf{x}$_{2}$&\textbf{x}$_{3}$&\textbf{x}$_{4}$&\textbf{b}&$\Sigma$\\ \hline
		
		1&0&-2&1&1&1\\ \hline
		0&1&-1&1&4&5\\ \hline
		0&0&$\boxed{1}$&1&6&8\\ \hline
		0&0&3&1&14&18\\ \hline
	\end{tabular}
\end{center}
3-ustundagi 3-elementni ruxsat etilgan element sifatida tanlab, jadvalni qayta yozamiz:
\begin{center}
	\begin{tabular}{|c|c|c|c|c|c|}
		\hline
		\textbf{x}$_{1}$&\textbf{x}$_{2}$&\textbf{x}$_{3}$&\textbf{x}$_{4}$&\textbf{b}&$\Sigma$\\ \hline
		
		1&0&0&3&14&18\\ \hline
		0&1&0&2&10&13\\ \hline
		0&0&1&1&6&8\\ \hline
		0&0&0&-2&-4&-6\\ \hline	
	\end{tabular}
\end{center}

4-satr elementlarini  $-2$ ga bo`lamiz:
\begin{center}
	\begin{tabular}{|c|c|c|c|c|c|}
		\hline
		\textbf{x}$_{1}$&\textbf{x}$_{2}$&\textbf{x}$_{3}$&\textbf{x}$_{4}$&\textbf{b}&$\Sigma$\\ \hline
		
		1&0&0&3&14&18\\ \hline
		0&1&0&2&10&13\\ \hline
		0&0&1&1&6&8\\ \hline
		0&0&0&$\boxed{1}$&2&3\\ \hline	
	\end{tabular}
\end{center}
4-satr 4-elementini ruxsat etilgan element sifatida tanlab jadvalni qayta yozamiz:
\begin{center}
	\begin{tabular}{|c|c|c|c|c|c|}
		\hline
		\textbf{x}$_{1}$&\textbf{x}$_{2}$&\textbf{x}$_{3}$&\textbf{x}$_{4}$&\textbf{b}&$\Sigma$\\ \hline
		
		1&0&0&0&8&9\\ \hline
		0&1&0&0&6&7\\ \hline
		0&0&1&0&4&5\\ \hline
		0&0&0&1&2&3\\ \hline
	\end{tabular}
\end{center}
Natijada tenglamalar sistemasi quyidagi ko`rinishga keladi:
$$\begin{cases}
	1\cdot x_{1}+0\cdot x_{2}+0\cdot x_{3}+0\cdot x_{4}=8\\
	0\cdot x_{1}+1\cdot x_{2}+0\cdot x_{3}+0\cdot x_{4}=6\\
	0\cdot x_{1}+0\cdot x_{2}+1\cdot x_{3}+0\cdot x_{4}=4\\
	0\cdot x_{1}+0\cdot x_{2}+0\cdot x_{3}+1\cdot x_{4}=2\\
\end{cases}$$
ya'ni, $x_{1}=8$, $x_{2}=6$, $x_{3}=4$, $x_{4}=2.\ \blacktriangle$
	\item Tenglamalar sistemasini yeching:
	
	$$\begin{cases}
		x_{1}+x_{2}-2x_{3}+x_{4}=1\\
		x_{1}-3x_{2}+x_{3}+x_{4}=0\\
		4x_{1}-x_{2}-x_{3}-x_{4}=1\\
		4x_{1}+3x_{2}-4x_{3}-x_{4}=2
	\end{cases}$$

$\triangle$ Jadvalni tuzib olamiz:
\begin{center}
\begin{tabular}{|c|c|c|c|c|c|}
	\hline
	$\boxed{1}$&1&-2&1&1&2\\ \hline
	1&-3&1&1&0&0\\ \hline
	4&-1&-1&-1&1&2\\ \hline
	4&3&-4&-1&-2&4\\ \hline
\end{tabular}
\end{center}
Ruxsat etilgan element sifatida 1-satr 1-ustunda turgan elementni tanladik. U holda:
\begin{center}
	\begin{tabular}{|c|c|c|c|c|c|}
		\hline
		1&1&-2&1&1&2\\ \hline
		0&-4&3&0&-1&-2\\ \hline
		0&-5&7&-5&-3&-6\\ \hline
		0&-1&4&-5&-2&-4\\ \hline	
	\end{tabular}
\end{center}
4-satr elementlarining ishoralarini o`zgartiramiz:
\begin{center}
	\begin{tabular}{|c|c|c|c|c|c|}
		\hline
		1&1&-2&1&1&2\\ \hline
		0&-4&3&0&-1&-2\\ \hline
		0&-5&7&-5&-3&-6\\ \hline
		0&$\boxed{1}$&-4&5&2&4\\ \hline
	\end{tabular}
\end{center}
Endi 4-satr 2-ustunda turgan element ruxsat etilgan element bo`ladi:
\begin{center}
	\begin{tabular}{|c|c|c|c|c|c|}
		\hline
		1&0&2&-4&-1&-2\\ \hline
		0&0&-13&20&7&14\\ \hline
		0&0&-13&20&7&14\\ \hline
		0&1&-4&5&2&4\\ \hline
	\end{tabular}
\end{center}
3-satr elementlaridan 2-satr elementlarini ayiramiz va 3-satrni o`chirib tashlaymiz:
\begin{center}
	\begin{tabular}{|c|c|c|c|c|c|}
		\hline
		1&0&2&-4&-1&-2\\ \hline
		0&0&-13&$\boxed{20}$&7&14\\ \hline
		0&1&-4&5&2&4\\ \hline
	\end{tabular}
\end{center}
2-satrning 4-elementi -- ruxsat etilgan element:
\begin{center}
	\begin{tabular}{|c|c|c|c|c|c|}
		\hline
	1&0&-0,6&0&0,4&0,8\\ \hline
	0&0&-13&20&7&14\\ \hline
	0&1&-0,75&0&0,25&0,5\\ \hline	
	\end{tabular}
\end{center}
Matritsa rangi  3 ga teng. Demak, tenglamalar sistemasida 3 ta  o`zaro bog`liq noma'lumlar: $x_{1}$, $x_{2}$, $x_{4}$ hamda bitta ozod noma'lum $x_{3}$  mavjud. Quyidagi tenglamalar sistemasini tuzamiz:
$$\begin{cases}
	1\cdot x_{1}+0\cdot x_{2}-0,6\cdot x_{3}+0\cdot x_{4}=0,4\\
	0\cdot x_{1}+0\cdot x_{2}-13\cdot x_{3}+20\cdot x_{4}=7\\ 
	0\cdot x_{1}+1\cdot x_{2}-0,75\cdot x_{3}+0\cdot x_{4}=0,25\\
\end{cases}$$
Demak, $x_{1}=0,4+0,6x_{3}$, $x_{2}=0,25+0,75u$, $x_{3}=u$, $x_{4}=0,35+0,65u$.
bu  yerda $u$ -- ixtiyoriy son. $\blacktriangle$

\item Tenglamalar sistemasini yeching:
$$\begin{cases}
	6x-5y+7z+8t=3\\
	3x+11y+2z+4t=6\\ 
	3x+2y+3z+4t=1\\
	x+y+z=0\\
\end{cases}$$
$\triangle$ Jadvalni tuzamiz:
\begin{center}
	\begin{tabular}{|c|c|c|c|c|c|}
		\hline
		6&-5&7&8&3&19\\ \hline
		3&11&2&4&6&26\\ \hline
		3&2&3&4&1&13\\ \hline
		$\boxed{1}$&1&1&0&0&3\\ \hline
	\end{tabular}
\end{center}
1-ustun 4-elementi -- ruxsat etilgan element.
\begin{center}
\begin{tabular}{|c|c|c|c|c|c|}
	\hline
	0&-11&$\boxed{1}$&8&3&1\\ \hline
	0&8&-1&4&6&17\\ \hline
	0&-1&0&4&1&4\\ \hline
	1&1&1&0&0&3\\ \hline
\end{tabular}
\end{center}
3-ustunning 1-elementi ruxsat etilgan element:
\begin{center}
	\begin{tabular}{|c|c|c|c|c|c|}
		\hline
		0&-11&$\boxed{1}$&8&3&1\\ \hline
		0&-3&0&12&9&18\\ \hline
		0&-1&0&4&1&4\\ \hline
		1&12&0&-8&-3&2\\ \hline
	\end{tabular}
\end{center}
3-satr elementlarining ishoralarini o`zgartiramiz:
\begin{center}
	\begin{tabular}{|c|c|c|c|c|c|}
		\hline
		0&-11&1&8&3&1\\ \hline
		0&-3&0&12&9&18\\ \hline
		0&$\boxed{1}$&0&-4&-1&-4\\ \hline
		1&12&0&-8&-3&2\\ \hline
	\end{tabular}
\end{center}
2-ustunning 3-elementi ruxsat etilgan element:
\begin{center}
	\begin{tabular}{|c|c|c|c|c|c|}
		\hline
		0&0&1&-36&-8&-43\\ \hline
		0&0&0&0&6&6\\ \hline
		0&$\boxed{1}$&0&-4&-1&-4\\ \hline
		1&0&0&40&9&50\\ \hline
	\end{tabular}
\end{center}
Natijada quyidagi tenglamalar sistemasini hosil qilamiz:
$$\begin{cases}
0\cdot x+0\cdot y+1\cdot z-36t=-8\\ 
0\cdot x+0\cdot y+0\cdot z-0\cdot t=6\\ 
0\cdot x+1\cdot y+0\cdot z-4\cdot t=-1\\
1\cdot x+0\cdot y+0\cdot z+40t=9\\
\end{cases}
$$
E'tibor berilsa, ikkinchi tenglamani $x$, $y$, $z$ va $t$ larning hech qanday qiymati qanoatlantirmaydi. Xulosa qilish mumkinki, berilgan tenglamalar sistemasi va biz hosil qilgan tenglamalar sistemasi bir-biriga ekvivalent emas. $\blacktriangle$

\item $A$ matritsa rangini Jordan-Gauss usulidan foydalanib aniqlang:
$$A=\begin{pmatrix}
	7&-1&3&5\\ 
	1&3&5&7\\
	4&1&4&6\\
	3&-2&-1&-1
\end{pmatrix}$$
$\triangle$ Jadval tuzamiz:
\begin{center}
	\begin{tabular}{|c|c|c|c|c|}
		\hline
		7&-1&3&5&14\\ \hline
		$\boxed{1}$&3&5&7&16\\ \hline
		4&1&4&6&15\\ \hline
		3&-2&-1&-1&-1\\ \hline
	\end{tabular}
\end{center}
Oxirgi ustunda mos satr elementlarining yig`indisi yozilgan. 1-ustunning 2-elementi ruxsat etilgan element:
\begin{center}
	\begin{tabular}{|c|c|c|c|c|}
		\hline
		0&-22&-32&-44&-98\\ \hline
		1&3&5&7&16\\ \hline
		0&-11&-16&-22&-49\\ \hline
		0&-11&-16&-22&-49\\ \hline
	\end{tabular}
\end{center}
1-satr elementlarini 2 ga bo`lamiz; 4- va 3-satr elementlaridan 1-satr elementlarini ayiramiz va 3- hamda 4-satrni o`chiramiz:
\begin{center}
	\begin{tabular}{|c|c|c|c|c|}
		\hline
		0&11&16&22&49\\ \hline
		1&3&5&7&16\\ \hline
	\end{tabular}
\end{center}
Hosil bo`lgan matritsaning ixtiyoriy ikkinchi tartibli determinanti noldan farqli. Demak $r(A)=2.\ \blacktriangle$
\end{enumerate}
Tenglamalar sistemasini Jordan-Gauss usulini qo`llagan holda yeching:
\begin{enumerate}\setcounter{enumi}{456}
	\item $\begin{cases}
		x_{1}+2x_{2}+x_{3}=8\\
		x_{2}+3x_{3}+x_{4}=15\\
		4x_{1}+x_{3}+x_{4}=11\\
		x_{1}+x_{2}+5x_{4}=23\\
		
	\end{cases}$
\inlineitem
$\begin{cases}
	x_{2}-x_{1}+x_{3}-x_{4}=-2\\
	x_{1}+2x_{2}-2x_{3}-x_{4}=-5\\
	2x_{1}-x_{2}-3x_{3}+2x_{4}=-1\\
	x_{1}+2x_{2}+3x_{3}+6x_{4}=-10\\
\end{cases}$

\item $\begin{cases}
	x_{1}+5x_{2}-2x_{3}-3x_{4}=1\\
	7x_{1}+2x_{2}-3x_{3}-4x_{4}=2\\
	x_{1}+x_{2}+x_{3}+x_{4}=5\\
	2x_{1}+3x_{2}2x_{3}-3x_{4}=4\\
	x_{1}-x_{2}-x_{3}-x_{4}=-2
\end{cases}$
\item Jordan-Gauss usulini qo`llagan holda matritsa rangini aniqlang:
$$A=\begin{pmatrix}
	1&2&3&1\\
	2&3&1&3\\
	3&1&2&5\\
	2&2&2&3
\end{pmatrix}$$
\end{enumerate}
