\chapter{Tekislikda analitik geometriya masalalari}

\section{To`g`ri chiziqli va qutb koordinatalari}

\textbf{1. To`g`ri chiziqda koordinatalarni belgilash. Kesmani berilgan nisbatda bo`lish}. $Ox$ koordinata o`qida yotgan, absissasi $x$ ga teng bo`lgan $M$ nuqtani $M(x)$ kabi belgilash qabul qilingan. 

Koordinata o`qidagi ikkita ixtiyoriy $M_{1}(x_{1})$ hamda $M_{2}(x_{2})$ nuqtalar orasidagi $d$ masofa:
\begin{equation}
	d=|x_{2}-x_{1}|
	\label{1}
\end{equation}
formula orqali aniqlanadi.

Ixtiyoriy to`g`ri chiziqda $AB$ kesma berilgan bo`lsin (A - kesma boshi, B-kesma oxiri). U holda to`g`ri chiziqda yotgan har qanday uchinchi C nuqta, AB kesmani $\lambda$ nisbatda bo`ladi, bu yerda $\lambda=\pm|AC|:|CB|$. Agar AC hamda CB kesmalar bir tomonga yo`nalgan bo`lsa, $\lambda$ ning ishorasi "+", agar kesmalar qarama-qarshi tomonga yo`nalgan bo`lsa "-" bo`ladi. Boshqacha aytganda, C nuqta A va B nuqtalar orasida yotganda $\lambda$ musbat, aks holda esa manfiy ishorani qabul qiladi. 

A va B nuqtalar $Ox$ koordinata o`qida yotgan bo`lsa, $A(x_{1})$ hamda $B(x_{2})$ nuqtalar hosil qilgan kesmani $\lambda$ nisbatda bo`ladigan C nuqtaning koordinatasi $C(\overline{x})$ quyidagi formula orqali aniqlanadi:
\begin{equation}
	\overline{x}=\frac{x_{1}+\lambda x_{2}}{1+\lambda}
	\label{2}
\end{equation}
Xususiy holda, $\lambda=1$ bo`lganda \eqref{2} formula kesma o`rtasining koordinatasini topish formulasiga aylanadi:
\begin{equation}
	\overline{x}=\frac{x_{1}+x_{2}}{2}
	\label{3}
\end{equation}
\begin{enumerate}
	\item To`g`ri chiziqda koordinatalari $A(3)$, $B(-2)$, $C(0)$, $D(\sqrt{2})$, $E(-3,5)$ nuqtalarni belgilang.
	\item AB kesma to`rtta nuqta orqali beshta teng bo`lakka bo`lingan. Agar $A(-3)$ va $B(7)$ bo`lsa, bo`linish nuqtasidan A  nuqtagacha bo`lgan eng qisqa masofani aniqlang.
	
	$\triangle$
	C($\overline{x}$) - izlanayotgan nuqtaning koordinatasi bo`lsin. U holda $\lambda=|AC|:|CB|=1/4$. \eqref{2} formulaga binoan,
	$$\overline{x}=\frac{x_{1}+\lambda x_{2}}{1+\lambda}=\frac{-3+(1/4)\cdot 7}{1+1/4}=-1,\ \ \ \textrm{ya'ni}\ \  C(-1).\blacktriangle$$
	\item $AB$ kesma uchlarining koordinatalari: $A(1)$ va $B(5)$. Bu kesmada yotmagan $C$ nuqta berilgan. 
	\item Nuqtalar orasidagi masofani aniqlang: 1) $M(3)$ va $N(-5)$; 2) $P(-11/2)$ va $Q(-5/2)$.
	\item Agar 1) $A(-6)$ va $B(7)$; 2) $C(-5)$ va $D(1/2)$ ma'lum bo`lsa, kesma o`rtasining koordinatalarini aniqlang.
	\item $P(2)$ nuqtaga nisbatan $N(-3)$ nuqtaga simmetrik bo`lgan $M$ nuqtani toping.
	\item $AB$ kesma ikkita nuqta yordamida teng uch bo`lakka bo`lingan. Agar $A(-1)$ va $B(5)$ bo`lsa, bo`linish nuqtalarining koordinatalarini aniqlang.
	\item $A(-7)$ va $B(3)$ nuqtalar berilgan. $AB$ kesmada yotmagan $C$ va $D$ nuqtalar uchun $|CA|=|BD|=0,5|AB|$ munosabat o`rinli bo`lsa, $C$ va $D$ nuqtalarning koordinatalarini toping.
\end{enumerate}

\textbf{2. Tekislikda to`g`ri burchakli koordinatalar sistemasi. Sodda misollar.}
Agar tekislikda to`g`ri burchakli $xOy$ Dekart koordinatalar sistemasi berilgan bo`lsa, ushbu tekislikdagi ixtiyoriy $M$ nuqtaning koordinatalari $M(x,y)$ ko`rinishida ifodalanadi.

$M_{1}(x_{1},y_{1})$ va $M_{2}(x_{2},y_{2})$ nuqtalar orasidagi masofa quyidagi formula orqali topiladi:
\begin{equation}
	d=\sqrt{(x_{2}-x_{1})^{2}+(y_{2}-y_{1})^{2}}
	\label{4}
\end{equation}
Xususiy hollarda, $M(x,y)$ nuqtadan koordinata boshigacha bo`lgan masofa quyidagi formula orqali topiladi:
\begin{equation}
	d=\sqrt{x^{2}+y^{2}}
	\label{5}
\end{equation}
$A(x_{1};y_{1})$ va $B(x_{2};y_{2})$ nuqtalar orqali hosil qilingan kesmani $\lambda$ nisbatda bo`luvchi $C(\overline{x};\overline{y})$ nuqtaning koordinatalari esa quyidagi formulalar yordamida hisoblanadi:
\begin{equation}
	\overline{x}=\frac{x_{1}+\lambda x_{2}}{1+\lambda};\ \ \ \ \ \overline{y}=\frac{y_{1}+\lambda y_{2}}{1+\lambda}
	\label{6}
\end{equation}
$\lambda=1$ bo`lgan xususiy hol uchun \eqref{6}-formula kesma o`rtasining koordinatasini topish formulasiga aylanadi:
\begin{equation}
	\overline{x}=\frac{x_{1}+x_{2}}{2};\ \ \ \ \ \overline{y}=\frac{y_{1}+y_{2}}{2}
	\label{7}
\end{equation}
Tomonlari $A(x_{1};y_{1})$, $B(x_{2};y_{2})$, $C(x_{3};y_{3})$ bo`lgan uchburchak yuzasi quyidagi formula orqali aniqlanadi:
\begin{equation}
	S=\frac{1}{2}\left| x_{1}(y_{2}-y_{3})+x_{2}(y_{3}-y_{1})+x_{3}(y_{1}-y_{2})\right|=
	\frac{1}{2}\left|(x_{2}-x_{1})(y_{3}-y_{1})-(x_{3}-x_{1})(y_{2}-y_{1})
	\right|
	\label{8}
\end{equation}
\eqref{8}-formulani soddaroq ko`rinishda quyidagicha ham yozish mumkin:
\begin{equation}
	S=\frac{1}{2}|\Delta|
	\label{9}
\end{equation}
bu yerda:
\begin{equation}
	\Delta=
	\begin{vmatrix}
		1&1&1\\
		x_{1}&x_{2}&x_{3}\\
		y_{1}&y_{2}&y_{3}\\
	\end{vmatrix}
\end{equation}\textit{uchinchi tartibli determinant haqida uchbu bobning 5-mavzusida batafsil ma'lumot berilgan.}

\begin{enumerate}
	\setcounter{enumi}{8}
	\item Koordinata tekisligida $A(4;3)$, $B(-2;5)$, $C(5;-2)$, $D(-4;-3)$, $E(-6;0)$, $F(0;4)$ nuqtalarni yasang.
	
	\item $A(3;8)$ va $B(-5;14)$ nuqtalar orasidagi masofani aniqlang.
	
	$\bigtriangleup$ \eqref{4}-formuladan foydalangan holda,
	$$d=\sqrt{(-5-3)^{2}+(14-8)^{2}}=\sqrt{\mathstrut 64+36}=10\ \ \ \blacktriangle$$
	
	\item Tomonlari $A(-3;3)$, $B(-1;3)$ va $C(11;-1)$ bo`lgan uchburchakning to`g`ri burchakli ekanligini isbotlang.
	
	$\bigtriangleup$ Uchburchak tomonlarining uzunliklarini hisoblaymiz:
	$$|AB|=\sqrt{\mathstrut (-1+3)^{2}+(3+3)^{2}}=\sqrt{\mathstrut 40}$$
	$$|BC|=\sqrt{(11+1)^{2}+(-1-3)^{2}}=\sqrt{160}$$
	$$|AC|=\sqrt{(11+3)^{2}+(-1+3)^{2}}=\sqrt{200}$$
	Shunday qilib, $|AB|^{2}=40$, $|BC|^{2}=160$, $|AC|^{2}=200$ bo`lsa, u holda $|AB|^{2}+|BC|^{2}=|AC|^{2}$. Ya'ni, uchburchak ikkita tomoni kvadratlarining yig`indisi uchinchi tomon uzunligining kvadratiga teng. Bundan ko`rinib turibdiki, $ABC$ uchburchak -- to`g`ri burchakli, $AC$ esa uning gipotenuzasi ekan. $\blacktriangle$
	
	\item $AB$ kesma uchlarining koordinatalari ma'lum: $A(-2;5)$ va $B(4;17)$. Ushbu kesmada yotgan $C$ nuqta kesmani $2:1$ nisbatda bo`ladi. $C$ nuqtaning koordinatalarini aniqlang.
	$\triangle$ Ma'lumki, $|AC|=2|CB|$, yoki $\lambda=|AC|:|CB|=2$. Bu yerda $x_{1}=-2$, $y_{1}=5$, $x_{2}=4$, $y_{2}=17$. Demak:
	$$\overline{x}=\frac{-2+2\cdot4}{1+2}=2;\ \ \ \ \ \overline{y}=\frac{5+2\cdot17}{1+2}=13.$$
	Javob: $C(2;13)$ $\blacktriangle$
	
	\item $C(2;3)$ nuqta -- $AB$ kesmaning o`rtasi. Agar $B(7;5)$ bo`lsa $A$ nuqtaning koordinatalarini aniqlang.
	
	$\triangle$ Masala shartiga ko`ra, $\overline{x}=2$, $\overline{y}=3$, $x_{2}=7$, $y_{2}=5$. Bu yerdan, $2=\frac{x_{1}+7}{2}$, hamda $3=\frac{y_{1}+5}{2}$. U holda,
	$x_{1}=-3$, $y_{1}=1$. Javob: $A(-3;1)$. $\blacktriangle$  
	
	\item $ABC$ uchburchak uchlarining koordinatalari berilgan: $A(x_{1};y_{1})$, $B(x_{2};y_{2})$, $C(x_{3};y_{3})$. Uchburchak medianalari kesishish nuqtasining koordinatalarini aniqlang.
	
	$\triangle$ $AB$ kesma o`rtasi -- $D$ nuqtaning koordinatalarini aniqlaymiz: $x_{D}=(x_{1}+x_{2})/2$ va $y_{2}=(y_{1}+y_{2})/2$. Medianalar kesishadigan $M$ nuqta, $CD$ kesmani $2:1$ nisbatda bo`ladi. U holda $M$ nuqtaning koordinatalari quyidagi formula orqali aniqlanadi:
	$$\overline{x}=\frac{x_{3}+2x_{D}}{1+2};\ \ \ \ \ \overline{y}=\frac{y_{3}+2y_{D}}{1+2},$$
	ya'ni
	$$\overline{x}=\frac{x_{3}+\frac{2(x_{1}+x_{2})}{2}}{3};\ \ \ \ \ \overline{y}=\frac{y_{3}+\frac{2(y_{1}+y_{2})}{2}}{3}$$
	Natijada $M$ nuqtaning koordinatalari uchun quyidagini hosil qilamiz:
	$$\overline{x}=\frac{x_{1}+x_{2}+x_{3}}{3};\ \ \ \ \ \overline{y}=\frac{y_{1}+y_{2}+y_{3}}{3}.\  \blacktriangle$$
	
	\item Uchlarining koordinatalari $A(-2;-4)$, $B(2;8)$ va $C(10;2)$ bo`lgan uchburchak yuzini hisoblang.
	
	$\triangle$ \eqref{8}-formuladan foydalangan holda hisoblaymiz:
	$$S=\frac{1}{2}\left|(2+2)\cdot(2+4)-(10+2)\cdot(8+4)\right|=\frac{1}{2}\left|24-144 \right|=60\ (\textrm{kv.birlik.})\ \blacktriangle$$ 
	
	\item Nuqtalar orasidagi masofani aniqlang: 1) $A(2;3)$ va $B(-10;-2)$; 2) $C(\sqrt{2};-\sqrt{7})$ va $D(2\sqrt{2};0)$.
	
	\item Uchlarining koordinatalari $A(4;3)$, $B(7;6)$ va $C(2;11)$ bo`lgan uchburchakning to`g`ri burchakli uchburchak ekanligini ko`rsating.
	
	\item Uchlarining koordinatalari $A(2;-1)$, $B(4;2)$ va $C(5;1)$ bo`lgan uchburchakning teng yonli uchburchak ekanligini ko`rsating.
	
	\item Uchburchak uchlarining koordinatalari: $A(-1;-1)$, $B(0;-6)$ va $C(-10;-2)$. Uchburchak $A$ uchidan o`tkazilgan mediana uzunligini toping.
	
	\item $AB$ kesma uchlarining koordinatalari berilgan: $A(-3;7)$ va $B(5;11)$. Ushbu kesma uchta nuqta orqali to`rtta teng qismga bo`lingan. Bo`linish nuqtalarining koordinatalarini toping.
	
	\item Uchlarining koordinatalari $A(1;5)$, $B(2;7)$ va $C(4;11)$ bo`lgan uchburchak yuzini hisoblang.
	
	\item Parallelogramm uchta ketma-ket kelgan uchlarining koordinatalari $A(11;4)$, $B(-1;-1)$ va $C(5;7)$ bo`lsa, uning to`rtinchi uchi koordinatalarini toping.
	
	\item Uchburchak ikkita uchining koordinatalari -- $A(3;8)$, $B(10;2)$ hamda medianalarining kesishish nuqtasi -- $M(1;1)$ berilgan. Ushbu uchburchak uchinchi uchining koordinatalarini aniqlang.
	
	\item Uchburchak uchlarining koordinatalari: $A(7;2)$, $B(1;9)$ va $C(-8;-11)$. Uchburchak medianalari kesishish nuqtasidan uning uchlarigacha bo`lgan masofalarni hisoblang.
	
	\item $L(0;0)$, $M(3;0)$ va $N(0;4)$ nuqtalar uchburchak tomonlari o`rtalarining koordinatalari bo`lsa, ushbu uchburchakning yuzini hisoblang.
\end{enumerate}	
\textbf{Qutb koordinatalari}. Tekislikda yotgan biron $M$ nuqtaning vaziyatini qutb koordinatalar sistemasida ifodalash uchun, ushbu nuqtadan koordinatalar boshigacha bo`lgan masofa $|OM|=\rho$ ($\rho$ -- nuqtaning radius-vektori) hamda $OM$ kesma va $Ox$ qutb o`qi orasidagi $\theta$ burchak ($\theta$ -- nuqtaning qutb burchagi) dan foydalaniladi. Agar burchak qutb o`qidan soat strelkasiga qarama-qarshi yo`nalishda hisoblansa,  $\theta$ ning qiymati musbat (+) bo`ladi. 

Agar $M$ nuqtaning qutb koordinatalar sistemasidagi koordinatalari $\rho>0$ va $0\ge\theta<2\pi$ mavjud bo`lsa, ushbu nuqta uchun qutb koordinatalar sistemasida cheksiz ko`p miqdordagi qutb koordinatalari juftliklari ($\rho; \theta+2k\pi$, bu yerda $k\in\mathbf{Z}$) mos keladi.

$M$ nuqtaning dekart ($x;y$) hamda qutb koordinatalari ($\rho;\theta$) orasida quyidagicha bog`lanish mavjud:
\begin{equation}
	x=\rho\cos\theta;\ \ \ \ \ y=\rho\sin\theta
	\label{10}
\end{equation}
\begin{equation}
	\rho=\sqrt{x^{2}+y^{2}};\ \ \ \ \ \textrm{tg}\theta=\frac{y}{x}
	\label{11}
\end{equation}

\begin{enumerate}
	\setcounter{enumi}{25}
	\item Qutb koordinatalar sistemasida quyidagi nuqtalarni yasang: $A(4;\ \pi/4)$, $B(2;\ 4\pi/3)$, $C(3;\ -\pi/6)$, $D(-3;\ \pi/3)$, $E(0;\ \alpha)$, $F(-1;\ -3\pi/4)$.
	
	\item $M(1;-\sqrt{3})$ nuqtaning qutb koordinatalarini toping.
	
	$\triangle$ \eqref{11}-formuladan foydalangan holda:
	$$\rho=\sqrt{1^{2}+\left(-\sqrt{3}\right)^{2}}=2;\ \ \ \ \ \textrm{tg}\theta=-\sqrt{3}$$
	Ko`rinib turibdiki, $M$ nuqta $IV$ chorakda yotadi. Demak, $\theta=5\pi/3$.
	Javob: $M(2;\ 5\pi/3)\  \blacktriangle$
	
	\item $A(2\sqrt{2};\ 3\pi/4)$ nuqtaning dekart koordinatalar sistemasidagi koordinatalarini toping.
	
	$\triangle$. \eqref{10}-formuladan foydalanib hisoblaymiz:
	$$x=2\sqrt{2}\cos{3\pi/4}=-2;\ \ \ \ \ y=2\sqrt{2}\sin{3\pi/4}=2$$
	Demak, $A(-2;2)\ \blacktriangle$
	
	\item $A(2\sqrt{3};\ 2)$, $B(0;\ -3)$, $C(-4;\ 4)$, $D(\sqrt{2};\ -\sqrt{2})$, $E(-\sqrt{2};\ -\sqrt{6})$, $F(-7;\ 0)$ nuqtalarning qutb koordinatalarini aniqlang.
	
	\item $A(10;\ pi/2)$, $B(2;\ 5\pi/4)$, $C(0;\ \pi/10)$, $D(1;\ -\pi/4)$, $E(-1;\ pi/4)$, $F(-1;\ -\pi/4)$ nuqtalarning to`g`ri burchakli (dekart) koordinatalarini aniqlang
	
	\item $M_{1}(\rho_{1};\theta_{1})$ va $M_{2}(\rho_{2};\theta_{2})$ nuqtalar orasidagi masofani hisoblang.
	
	$\bullet$ $OM_{1}M_{2}$ uchburchak uchun kosinuslar teoremasini qo`llang.
	
	\item $M(3; \ \pi/4)$ va $N(4; \ 3\pi/4)$ nuqtalar orasidagi masofani hisoblang.
	
	\item Qutb o`qiga nisbatan $M(\rho,\ \theta)$ nuqtaga simmetrik bo`lgan nuqtaning qutb koordinatalarini toping.
	
	\item Tekislikka nisbatan $M(\rho,\ \theta)$ nuqtaga simmetrik bo`lgan nuqtaning qutb koordinatalarini toping.
	
	\item $(3;\ \pi/6)$, $(5;\ 2\pi/3)$, $(2;\ -\pi/6)$ nuqtalarga: 1) tekislikka nisbatan; 2) qutb o`qiga nisbatan simmetrik bo`lgan nuqtalarning koordinatalarini toping.
	
	\item Qutb o`qiga perpendikulyar tekislik orqali o`tuvchi chiziqqa nisbatan $M(\rho;\ \theta)$ nuqtaga simmetrik bo`lgan nuqta koordinatalarini toping.		
\end{enumerate}

\textbf{4. Chiziq tenglamalari}. 

\begin{enumerate}
	\setcounter{enumi}{36}
	\item Kesmaning bir uchi absissa o`qiga tutashadi, ikkinchi uchi esa ordinata o`qiga. Agar kesma uzunligi $c$ bo`lsa,shu kesma o`rtasini ifodalaydigan chiziq tenglamasini toping.
	
	$\triangle$ $M(x;\ y)$ kesmaning o`rtasi bo`lsin. $OM$ kesmaning uzunligi
\end{enumerate}







\section{To`g`ri chiziq}
\textbf{To`g`ri chiziqning umumiy tenglamasi}.
