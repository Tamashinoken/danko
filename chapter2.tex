\chapter{Vektor algebrasi elementlari}
\section{Fazoda to`g`ri burchakli koordinatalar}
\hspace{0.6cm}Fazoda to`g`ri burchakli $Oxyz$ Dekart koordinatalar sistemasi berilgan bo`lsin. Ushbu fazodagi ixtiyoriy $M$ nuqtaning koordinatalari $M(x,y,z)$ ko`rinishida ifodalanadi. Bunda $x$ - absissa, $y$ - ordinata, $z$-applikata deyiladi.

$A_{1}(x_{1},y_{1}, z_{1})$ va $A_{2}(x_{2},y_{2}, z_2)$ nuqtalar orasidagi masofa quyidagi formula orqali topiladi:
\begin{equation}
	d=\sqrt{(x_{2}-x_{1})^{2}+(y_{2}-y_{1})^{2}(z_2-z_1)^2}
	\label{2.1}
\end{equation}
Xususiy hollarda, $M(x,y,z)$ nuqtadan koordinata boshigacha bo`lgan masofa quyidagi formula orqali topiladi:
\begin{equation}
	d=\sqrt{x^{2}+y^{2}+z^2}
	\label{2.2}
\end{equation}
Agar $A(x_1;y_1;z_1)$ va $B(x_2;y_2;z_2)$ nuqtalar bilan chegaralangan kesma $C(\overline x;\overline y; \overline z)$ nuqta orqali $\lambda$ nisbatda bo`linsa, $C$ nuqtaning koordinatalari quyidagi formula orqali topiladi:
\begin{equation}
	\overline x=\frac{x_{1}+\lambda x_2}{1+\lambda};\ \ \ \overline y=\frac{y_{1}+\lambda y_2}{1+\lambda};\ \ \ \overline z=\frac{z_{1}+\lambda z_2}{1+\lambda};
	\label{2.3} 
\end{equation}
Xususiy holda, kesma o`rtasining koordinatalarini topish formulasi quyidagicha bo`ladi:
\begin{equation}
	\overline x=\frac{x_1+x_2}{2};\ \ \ \overline y=\frac{y_1+y_2}{2};\ \ \ \overline z=\frac{z_1+z_2}{2}
	\label{2.4}
\end{equation}

\begin{enumerate} \setcounter{enumi}{230}
	
	\item $M_1(2; 4; -2)$ va $M_2(-2; 4; 2)$ nuqtalar berilgan. $M_1M_2$ kesmani  $\lambda=3$ nisbatda bo`ladigan $M$ nuqtaning koordinatalarini aniqlang. 
	
	$\triangle$ \eqref{2.3}-formuladan foydalanamiz:
	\begin{multline*}
		x_M=\frac{x_1+\lambda x_2}{1+\lambda}=\frac{2+3(-2)}{1+3}=-1;\ \ \ y_M=\frac{y_1+\lambda y_2}{1+\lambda}=\frac{4+3\cdot4}{1+3}=4;\\ \ \ z_M=\frac{z_1+\lambda z_2}{1+\lambda}=\frac{-2+3\cdot2}{1+3}=1
	\end{multline*}
	
	Demak, $M(-1;4;1).\ \blacktriangle$
	
	\item $A(1;1;1)$, $B(5;1;2)$, $C(7;9;1)$  nuqtalardan tuzilgan uchburchak berilgan. $A$ burchak bissektrissasining $CB$ tomon bilan kesishish nuqtasi $D$ nuqta koordinatalarini aniqlang.
	
	$\triangle$ Uchburchak tomonlarining uzunliklarini hisoblaymiz:
	
	$$|AC|=\sqrt{(x_C-x_A)^2+(y_C-y_A)^2+(z_C-z_A)^2}=\sqrt{(7-1)^2+(9-1)^2+(1-1)^2}=10$$
	$$|AB|=\sqrt{(x_B-x_A)^2+(y_B-y_A)^2+(z_B-z_A)^2}=\sqrt{(5-1)^2+(1-1)^2+(-2-1)^2}=5$$
	
	Ko`rinib turibdiki, $|CD|:|DB|=10:5=2$. Ya'ni bissektrissa $CB$ tomonni qo`shni tomonlar uzunliklariga proporsional nisbatda bo`ladi. Shunday qilib, 
	
	$$x_D=\frac{x_C+\lambda x_B}{1+\lambda}=\frac{7+2\cdot5}{1+2}=\frac{17}{3};\ \ \ \ y_D=\frac{y_C+\lambda y_B}{1+\lambda}=\frac{9+2\cdot1}{1+2}=\frac{11}{3};\ \ \ \ z_D=\frac{z_C+\lambda z_B}{1+\lambda}=\frac{1+2(-2)}{1+2}=-1.\ \blacktriangle$$
	
	\item $Ox$ o`qida $A(2;-4;5)$ va $B(-3;2;7)$ nuqtalardan bir xil masofada joylashgan nuqtani aniqlang.
	
	$\triangle$ $M$ -- qidirilayotgan nuqta bo`lsin. Ushbu nuqta uchun $|AM|=|MB|$ tenglik bajarilishi kerak. $M$ nuqta $Ox$ o`qda yotishini hisobga olsak, uning koordinatalari $(x;0;0)$ bo`ladi. U holda:
	$$|AM|=\sqrt{(x-2)^2+(-4)^2+5^2},\ \ \ \ |MB|=\sqrt{(x+3)^2+2^2+7^2}$$
	Bu tengliklarni kvadratga oshirib quyidagini hosil qilamiz:
	$$(x-2)^2+41=(x+3)^2+53,\ \mbox{yoki}\ \ 10x=-17\Rightarrow x=-1,7$$
	Shunday qilib qidirilayotgan nuqtaning koordinatalari $M(-1,7;\ 0;\ 0).\ \blacktriangle$
	
	\item $A(3;3;3)$ va $B(-1;5;7)$ nuqtalar berilgan. $AB$ kesmani teng uch bo`lakka bo`luvchi $C$ va $D$ nuqtalarning koordinatalarini toping.
	
	\item Uchburchak uchlarining koordinatalari: $A(1;2;3)$, $B(7;10;3)$ va $C(-1;3;1)$. $A$ burchakning o`tmas burchak ekanligini ko`rsating.
	
	\item Uchlarining koordinatalari $A(2;3;4)$, $B(3;1;2)$ va $C(4;-1;3)$ bo`lgan uchburchak og`irlik markazining\footnote{Uchburchak medianalarining kesishish nuqtasi} koordinatalarini toping. 
	
	\item $A(3;1;4)$ va $B(-4;5;3)$ nuqtalardan bir xil uzoqlikda joylashgan $M$ nuqta,  koordinata boshidan $C(0;6;0)$ nuqtagacha bo`lgan kesmani qanday nisbatlarda bo`ladi?
	
	\item $Oz$ o`qida $M_1(2;4;1)$ va $M_2(-3;2-5)$ nuqtalardan bir xil uzoqlikda joylashgan nuqtani aniqlang.
	
	\item $xOy$ tekislikda $A(1;-1;5)$, $B(3;4;4)$ va $C(4;6;1)$ nuqtalardan bir xil uzoqlikda joylashgan nuqta koordinatalarini toping. 
\end{enumerate}

\section{Vektorlar va ular ustida amallar}



\section{Skalyar va vektor ko`paytma. Aralash ko`paytma}


